\chapter{Expander graphs}
\label{mathchapter}

Graphs appear everywhere in mathematics but they are especially common in 
combinatorics. 

\begin{definition}
(Graph) A graph is an ordered pair $G=(V,E)$. The elements of $V$ are called 
vertices. The elements of $E$ are 2-element subsets of $V$ and they are called 
the edges of the graph. 
\end{definition}

The connectivity of graph is the minimum number of vertices or edges that are 
required to be removed in order to disconnect the graph. However, this often 
means that one of the components is very small. A more nuanced view is provided 
by separators:

\begin{definition}
(Separator) A separator is a partition of the vertices of a graph into sets 
$(A,B,C)$ where $|A|, |B| \geq c|V|$ where $0<c<1/2$ is called the separation 
constant. The size of the separator is $|C|$.  
\end{definition}


The main objects of study in this work are polytopes. They are interesting 
objects in their own right but also play an important role in e.g. 
optimization. In the following section we introduces the basics of polytopes. 
More in-depth knowledge can be found in Ziegler's book [citation]. There are 
two ways to define a polytope, the vertex (or interior) representation and the 
half-space (or exterior) representation.

\begin{definition}
(V-Polytope) A polytope is the convex hull of a finite number of points in 
$\mathbb{R}^d$, that is 
\begin{equation}
 P = \left\{x\in \mathbb{R}^d \middle| x = \sum_{i=1}^{|V|} \lambda_i v_i, 
0\leq \lambda_i \leq 1, \sum_{i=1}^{|V|} \lambda_i = 1 \right\}
\end{equation}

\end{definition}

\begin{definition}
 (H-polytope) A polytope is the bounded intersection of a finite number of 
half-space, that is
\begin{equation}
 P = \bigcap_{i=1}^n \{x \in \mathbb{R}^d \mid \mathbf{a}_i^T \mathbf{x} \leq 
c_i \}
\end{equation}
where $a_i \in \mathbb{R}^d$ are normal vectors of the hyperplanes and $c_i \in 
mathbb{R}$ are displacements. If this description is minimal with respect to 
the number of half-spaces required to describe $P$, it is called irredundant.
\end{definition}

It is a fundamental theorem of polytopes that these two representations are 
equivalent.

The dimension of a polytope is the dimension of its affine hull. The affine 
hull of a set of points is the smallest hyperplane which contains all of the 
points. If the lowest dimensional hyperplane which contains $n$ points is 
$(n-)1$-dimensional, the points are set to be affinely independent. The 
boundary 
of a polytope consists of lower dimensional polytopes called faces. 
\begin{definition}
 (Face) A face is an intersection of a polytope with a hyperplane 
\begin{equation}
 F = P \cap \{x \in \mathbb{R}^d \mid \mathbf{a}\mathbf{x} = c\}
\end{equation}
where the hyperplane has to be \textit{valid}, that is $\mathbf{a}\mathbf{x} 
\leq c$ holds for all points in $P$. 

\end{definition}

Some faces have special names: the 0-dimensional faces are called 
\textit{vertices}, the 1-dimensional faces are \textit{edges} and the 
$d-1$-dimensional faces are \textit{facets}. The faces of various dimensions 
are counted by the face vector
\begin{definition}
 (Face vector) The face vector is $(f_0, f_1, \dots, f_{d-1})$ where $f_i$ is 
the number of $i$-dimensional faces.
\end{definition}
 The faces of a polytope 
form a lattice with ordering given by inclusion. This lattice is called the 
\textit{face lattice} of the polytope.

A polytopal complex is a collection of polytopes that fit together nicely.
More precisely:
\begin{definition}
 (Polytopal complex) A polytopal complex $\mathfrak{C}$ is a finite set of 
polytopes with the following properties:
\begin{enumerate}
 \item The empty polytope $\emptyset \in \mathfrak{C}$.
  \item If $F \subset P$ is a face of $P$ and $P \in \mathfrak{C}$, then 
$F\mathfrak{C}$.
\item The intersection $P\cap Q$ of $P,Q \in \mathfrak{C}$ is a face of both 
and $P\cap Q \in \mathfrak{C}$.
\end{enumerate}

\end{definition}

Examples of polytopal complexes include the polytope itself and its 
\texit{boundary complex} which is the collection of all proper faces of $P$. 
The $k$-skeleton of a polytope is the collection of all faces of dimension 
$\leq k$. Hence the $d$-skeleton of a $d$-polytope is the polytope itself and 
the boundary complex is the $(d-1)$-skeleton. The 1-skeleton of a polytope is 
called its graph and will be important for us later on.

Each polytope has a dual polytope associated to it. Its face lattice is the 
lattice of the original polytope ``upside down''. Hence its $i$-faces 
correspond to the $(d-i)$-dimensional faces of the original. The dual of the 
dual is the original polytope. 

In any dimension, the polytope with the fewest vertices is the simplex. 
\begin{definition}
 (Simplex) A $d$-simplex is the convex hull of $d+1$ affinely independant 
points.
\end{definition}
The faces of a simplex are lower dimensional simplices. A polytope whose facets 
(but not necessarily the polytope itself) are simplices is called 
\textit{simplicial}. The dual of a simplicial polytope is called 
\textit{simple}. The vertices of a simple $d$-polytope are contained in exactly 
$d$ edges and $d$ facets.

Two polytopes $P\subset \mathbb{R}^d, Q \subset \mathbb{R}^e$ are 
\textit{affinely isomorphic} if there is an affine map $f: \mathbb{R}^d 
\rightarrow \mathbb{R}^e$ which is a bijection between the points of the 
polytope. A more relaxed equivalence relation is combinatorial equivalence. Two 
polytopes are combinatorially equivalent if their face lattices are isomorphic 
i.e. there is an bijection between the elements of the lattice which respects 
the inclusion relation. 

Cutting of a vertex is a way to get a new polytope from an old one. This is 
done by adding a new inequality to the description. This inequality is not 
valid for the vertex which is cut off but is valid for all other vertices. All 
the polytopes formed this way are combinatorially equivalent (but not 
necessarily affinely isomorphic). For each face intersected by the hyperplane, 
a face of one dimension lower is added to the face lattice. The dual operation 
is called stacking on a facet. It means adding a new vertex to the polytope 
which satisfies all the inequalities in the irredundant description with strict 
inequality except one (which is violated).



\section{The End}
\label{sec:end}

Finally, this is the end.  The bibliography starts on
the next page.
Note how the \verb2\hyperref2 package
(mentioned in chapter \ref{introchap})
also makes hyperlinks from references
(e.g., Mulick\cite{mulick})
to entries in the bibliography.

