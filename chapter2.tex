\chapter{Introduction}
\label{intro}

A separator is a partition of the vertices of a graph into three sets. There are two "large" sets $A$ and $B$ with no edges
between a vertex in $A$ and a vertex in $B$, and 
an additional set $C$ containing the rest of the vertices. As moving vertices from $A$ or $B$ to
$C$ preserves the separation property, we are interested in the minimum size of the set $C$. 
This is called the size of the separator. A similar definition exists for edge separators, but we study
only regular graphs where the difference between the two doesn't matter.
Separators are closely connected to the expander graphs; a family of expander graph has minimal separators
whose size is linear in the number of vertices.



The story of separators of polytopes begins with Ungar \cite{Ungar1951},
who proved that planar graphs, and therefore graphs of 3-polytopes, have minimal 
separators of size $\Omega(\sqrt{n} \log n)$. Lipton and Tarjan proved the planar 
separator theorem  \cite{LiTa}, which gives the correct bound
$\Omega(\sqrt{n})$. This result is best possible, since the $m\times m$ grid graph
can't be separated by fewer than $\Omega(m)$ vertices. Miller and Thurston \cite{MillerThurston-separators} gave a 
geometric proof of the planar separation theorem using the  
Koebe–Andreev–Thurston circle-packing theorem.

Miller, Teng, Thurston and Vavasis \cite{MillerTengThurstonVavasis} generalized 
the theorem to intersection graphs of ball packings in $d$ dimensions. This 
led Kalai (\cite{Kal97}, corrected in \cite{kalai04:_polyt}) to conjecture that 
simple $d$-polytopes would also have small separators.
\begin{conjecture}
\label{simpleconjecture}
Simple $d$-polytopes have separators of size
$\Omega \left(n^{\frac{d-2}{d-1}}\right)$.
\end{conjecture}
Simplicity is necessary as the graph of the cyclic polytopes is complete for 
$d\geq 4$ and hence has no vertex separator. Kalai also refers to 
\cite{MillerTengThurstonVavasis} for the claim that there are triangulations of 
$S^3$ which don't have separators the size of $\Omega(n/\log n)$ but this is not 
stated in the paper. 

\chapter{Graphs and Polytopes}
\label{mathchapter}

\section{Polytopes}

The main objects of study in this work are polytopes and their graphs. Polytopes are 
interesting 
objects in their own right, but also play an important role in
optimization. In the following section we introduce the basics of polytopes. 
A more thorough treatment can be found in books by Ziegler \cite{PolyLec} 
and Gr\"unbaum \cite{Grunbaum69convexpolytopes}. 

There are 
two ways to define a polytope, the vertex (or interior) representation and the 
half-space (or exterior) representation.

\begin{definition}
(V-Polytope) A polytope is the convex hull of a finite number of points $V 
\subset \mathbb{R}^d$, that is 
\begin{equation}
 P = \left\{x\in \mathbb{R}^d \middle| x = \sum_{i=1}^{|V|} \lambda_i v_i, 
0\leq \lambda_i \leq 1, \sum_{i=1}^{|V|} \lambda_i = 1 \right\}.
\end{equation}

\end{definition}

\begin{definition}
 (H-polytope) A polytope is the bounded intersection of a finite number of 
half-space, that is
\begin{equation}
 P = \bigcap_{i=1}^n \left\{x \in \mathbb{R}^d \mid \mathbf{a}_i^T \mathbf{x} 
\leq 
c_i \right\}
\end{equation}
where $\mathbf{a}_i \in \mathbb{R}^d$ are normal vectors of the hyperplanes and 
$c_i \in 
\mathbb{R}$ are displacements. If this description is minimal with respect to 
the number of half-spaces required to describe $P$, it is called 
\textit{irredundant}.
\end{definition}

It is a fundamental theorem of polytopes that these two representations are 
equivalent.
\begin{theorem}
(Minkowski-Weyl Theorem, \cite[Thm. 1.1.]{PolyLec}) Let $P \subset \mathbb{R}^d$. 
$P$ is a $H$-polytope if and only if $P$ is a $V$-polytope.
\end{theorem}
The dimension of a polytope is the dimension of its affine hull. The \textit{affine 
hull} of a set of points is the dimension of the smallest hyperplane which 
contains all of the points. A $d$-polytope in $\mathbb{R}^d$ is said to be 
\textit{full-dimensional}. If the lowest dimensional hyperplane which contains 
$n$ points is $(n-1)$-dimensional, the points are set to be \textit{affinely 
independent}.

 The boundary 
of a polytope consists of lower dimensional polytopes called faces. 
\begin{definition}
 (Face) A face is an intersection of a polytope with a hyperplane 
\begin{equation}
 F = P \cap \{x \in \mathbb{R}^d \mid \mathbf{a}\mathbf{x} = c\}
\end{equation}
where the hyperplane has to be \textit{valid}, that is $\mathbf{a}\mathbf{x} 
\leq c$ holds for all points in $P$. 

\end{definition}
Both the empty set and $P$ are faces of $P$, they are given by $0x \le 1$ and $0x \le 0$, respectively.
The empty set is defined to have dimension -1.
A face that is not equal to all of $P$ is called a \textit{proper face}.
Some faces have special names: the 0-dimensional faces are called 
\textit{vertices}, the 1-dimensional faces are called \textit{edges} and the 
$d-1$-dimensional faces are called \textit{facets}.
 The faces of a polytope 
form a lattice with ordering given by inclusion. This lattice is called the 
\textit{face lattice} of the polytope.

The faces of various dimensions 
are counted by the face vector.
\begin{definition}
 (Face vector) The face vector is $(f_0, f_1, \dots, f_{d-1}) \in \mathbb{Z}^d$ 
where $f_i$ is 
the number of $i$-dimensional faces.
\end{definition}
The face vector satisfies the famous Euler-Poincare equation
\begin{equation}
 \sum_{i=0}^{d-1} (-1)^i f_i = 1 - (-1)^d.
\end{equation}
This is the only linear equation satisfied by the face vector. More 
constraints can be found by studying the \textit{flag vector}.

\begin{definition}
 (Flag vector) The flag vector counts the chains of faces in specified 
dimensions.
\end{definition}

For example, the entry $f_{013}$ counts the number of chains in the face 
lattice which has entries in dimensions $(0,1,3)$. The complete face vector has $2^d$ entries,
but many of them are redundant. In particular, in 4 dimensions whrere we work in,
knowing the vector $(f_0, f_1, f_2, f_3, f_{03})$ is enough to figure out the rest of the entries. 

Now that we have defined what a polytope is, it we need a criterion to decide which polytopes 
count as "the same". There are two ways to do this, with one stricter than the other.
Two polytopes $P\subset \mathbb{R}^d, Q \subset \mathbb{R}^e$ are 
\textit{affinely isomorphic} if there is an affine map $f: \mathbb{R}^d 
\rightarrow \mathbb{R}^e$ which is a bijection between the points of the 
polytope. A more relaxed equivalence relation is combinatorial equivalence. Two 
polytopes are \textit{combinatorially equivalent} if their face lattices are isomorphic 
i.e.\ there is an bijection between the elements of the lattice which respects 
the inclusion relation. Two combinatorially equivalent polytopes are said to have the 
same \textit{combinatorial type}. 

We are interested in studying the graphs of polytopes. The most theoretically 
pleasing way to define them is to consider them a special class of 
\textit{polytopal complexes}.
A polytopal complex is a collection of polytopes that fit together nicely.
More precisely:
\begin{definition}
 (Polytopal complex) A polytopal complex $\mathfrak{C}$ is a finite set of 
polytopes with the following properties:
\begin{enumerate}
 \item The empty polytope $\emptyset \in \mathfrak{C}$.
  \item (Downward closure) If $F \subset P$ is a face of $P$ and $P \in 
\mathfrak{C}$, then 
$F \in \mathfrak{C}$.
\item The intersection $P\cap Q$ of $P,Q \in \mathfrak{C}$ is a face of both.

\end{enumerate}

\end{definition}

% Examples of polytopal complexes include the polytope itself and its 
% \textit{boundary complex} which is the collection of all proper faces of $P$. 
Any polytope $P$ with its faces is a polytopal complex, as is the 
\textit{boundary complex} of 
$P$. The boundary complex is the collection of proper faces of $P$.
The $k$-skeleton of a polytope is the collection of all faces of dimension $\leq 
k$. Hence the $d$-skeleton of a $d$-polytope is the polytope itself and the 
boundary complex is the $(d-1)$-skeleton. The graph of the polytope is its 
1-skeleton and it is denoted by $G(P)$.

% Each polytope has a dual polytope associated to it. Its face lattice is the 
% lattice of the original polytope ``upside down''. Hence its $i$-faces 
% correspond to the $(d-i)$-dimensional faces of the original. The dual of the 
% dual is the original polytope. 

Two polytopes are said to be dual to each other if there is an inclusion 
reversing bijection between their face lattices. Clearly, the dual of a 
$d$-polytope is another $d$-polytope. The fact that every polytope has a dual 
can be seen from the \textit{polar set}. Let $A\subset \mathbb{R}^d$, then the 
polar set $A^*$ is defined by
\begin{equation}
 A^* = \{y \in \mathbb{R}^d \mid \langle x,y\rangle \leq 1 \, \forall  x  \in A 
\} 
\end{equation}
If $A$ is a convex compact set with $0 \in$ int $A$, so is $A^*$. In 
particular, if $A$ is a polytope containing $0$ in the interior, $A^*$ is dual 
to it. 

After all these theoretical construction, let's have a look at some concrete examples.
In any dimension, the polytope with the fewest vertices is the simplex. 
\begin{definition}
 (Simplex) A $d$-simplex is the convex hull of $d+1$ affinely independant 
points.
\end{definition}
The faces of a simplex are lower dimensional simplices and any $k\le d+1$ vertices
of the simplex form a face. 
 A polytope whose facets 
(but not necessarily the polytope itself) are simplices is called 
\textit{simplicial}. The dual of a simplicial polytope is called 
\textit{simple}. The vertices of a simple $d$-polytope are contained in exactly 
$d$ edges and $d$ facets. 

Another important polytope is the $n$-cube.

\begin{definition}
 The $m$-cube $Q^m$ is a simple polytope defined by the inequalities
\begin{equation}
 0 \leq x_i \leq 1 
\end{equation}
for $i \in 1,\dots, m$.
\end{definition}
The $m$-cube has $2^m$ vertices, $m2^{m-1}$ edges and $2m$ facets which
are $(m-1)$-cubes. Each vertex $v$ of the $m$-cube has coordinates equal to 0 
or 1, that is $v\in \{0,1\}^m$. The coordinate vector is called the 
\textit{label} of the vertex. Two vertices of the cube are adjacent if and only 
if their labels differ in exactly one coordinate. A polytope whose facets are combinatorial cubes
is called cubical.


Vertex truncation is a common method of creating new polytopes from old ones.
We will use it extensively.
\begin{definition}
(Vertex truncation) Let $P$ be a polytope with vertex set $V$. Truncating a vertex $v \in V$ means adding
a new inequality to the halfspace description, which is violated by $v$ but valid for all the others vertices in $V$.
\end{definition}
All the polytopes formed this way are combinatorially equivalent (but generally not affinely isomorphic). 
For each face intersected by the hyperplane, 
a face of one dimension lower is added to the face lattice. The new facet is 
called the \textit{vertex figure} of the vertex $v$. The vertex figure's combinatorial type 
is depends only on the choice of vertex, not on the inequality. Vertex cuts can be generalized to cutting of 
faces of higher dimensions by requiring the added inequality to be violated by 
all the vertices in the face. 

The dual operation of truncation is called stacking. 
It means adding a new vertex to the polytope which satisfies all the 
inequalities in the irredundant description with strict inequality except the 
ones corresponding to the face being stacked on.

Another way to produce new polytopes is taking prisms. This operation takes a 
$d$-polytope $P$ in $\mathbb{R}^d$ and produces a $(d+1)$-polytope in 
$\mathbb{R}^{d+1}$. 
\begin{definition}
 A prism is the convex hull of a copy of $P$ embedded on the hyperplane 
$x_{d+1} = 0$ and another copy on the the hyperplane $x_{d+1} = 1$.
\end{definition}

 The new polytope prism$(P)$ has 2 copies of 
each face at the top and bottom and for each face $F$ a face isomorphic to 
prism$(F)$ in the middle. Two vertices in $G(\text{pyr}(P))$ are adjacent if and only if
they are copies of the same vertex or if they are in the same copy and are 
adjacent there. 
The $n$-cube can be defined as the $n$-fold 
prism over the zero dimensional polytope (a point). The dual construction to prism 
is the bipyramid. 
\begin{definition}
 (Bipyramid) Let $P\subset \mathbb{R}^d$ be a $d$-polytope with zero in the 
interior. Then the bipyramid bipyr$(P)$ is the convex hull of $P$ embedded on 
the hyperplane $x_{d+1} = 0$ and the points $(0,0,\dots, 1)$ and $(0,0,\dots, 
-1)$.
\end{definition}

\section{Graphs}

Graphs appear everywhere in mathematics but they are especially common in 
combinatorics. 

\begin{definition}
(Graph) A graph is an ordered pair $G=(V,E)$. The elements of $V$ are called 
vertices. The elements of $E$ are 2-element subsets of $V$. They are called 
the edges of the graph. 
\end{definition}
A graph is often visualized by connecting dots (vertices) with lines (edges) on a plane. 
If such a drawing can be made without any lines intersecting, the graph is called
planar.
We consider only simple graphs, i.e.\ graphs without loops or parallel 
edges.

If $(u,v) \in E$, the vertices $u$ and $v$ are said to be \textit{adjacent}. 
The vertices $u$ and $v$ are \textit{incident} to the edge $(u,v)$. 
\begin{definition}
 (Path) A set of edges $(u_1, u_2), (u_2, u_3), \dots (u_{k-1}, u_k) \subset E$ 
is called a path between the vertices $u_1$ and $u_k$ if all the $u_i$ are 
distinct.
\end{definition}
If there is a path between every pair of vertices, the graph is said to be 
\textit{connected}. Otherwise it is \textit{disconnected}. A disconnected graph 
consists of several \textit{connected components}.

The \textit{degree} of a vertex is the number of edges it is incident to. 
If all the vertices in the graph have degree $d$, the graph is called 
$d$-regular. A graph is called is $k$-vertex-connected if removal of fewer than 
$k$ vertices does not disconnect the graph. Edge connectivity is defined in an 
analogous manner. Steinitz' theorem states that the graphs of 
3-polytopes are precisely all planar, simple, 3-vertex-connected graphs. Hence 
removal of only a few vertices disconnects the graph. However, this
often means that one of the components is very small. A more nuanced view is 
provided 
by separators:

\begin{definition}
(Separator) A (vertex) separator with separation constant $0<c<1/2$ is a 
partition of the vertices of a graph into 
sets $(A,B,C)$  with $cn \le |A| \le |B| \le (1-c)n$ so that there is no edge between 
a vertex in $A$ and a vertex in $B$. The \textit{size} of the separator
separator is $|C|$.  
\end{definition}

Another way to think about separators is by minimizing \textit{neighborhoods} 
of a set of given size.

\begin{definition}
 (Neighborhood) Let $S \subset V$ be a set of vertices. Then its neighborhood 
$N(S)$ is the set $\{v \in V\setminus S \mid (u,v) \in E, u \in S\}$. 
\end{definition}

We are generally interested how the size of the smallest separator behaves as a 
function 
of the number of vertices for some fixed graph family.

Separators can also be defined by removing edges instead of vertices. However, 
for regular graphs the sizes of these separators differ by a constant multiple 
so we ignore the distinction.

The size of minimal separators in the graphs of 3-polytopes was discovered by 
Lipton and Tarjan \cite{LiTa} in 1979. They showed that planar graphs have 
separators of size $O(\sqrt{n})$ for separation constant $\nicefrac{1}{3}$. This is 
the best possible result since an $m\times m$-grid requires $m$ vertices to be 
removed to be separated. However, in higher dimensions such results are 
impossible due to the existance of \textit{neighborly polytopes}.

\begin{definition}
 A $k$-neighborly polytope on $n$-vertices is a $d$-polytope with the $k$-skeleton of a $(n-1)$-simplex.
A $\left\lfloor \frac{d}{2} \right\rfloor$-neighborly $d$-polytope is simply called neighborly, as this is the maximal
neighborliness of a polytope which is not a simplex.
\end{definition}

Any neighborly polytope of dimension $\geq 4$ has the complete graph as 
its graph and can't be separated by removing vertices. Simple 
$d$-polytopes on the other hand have $d$-regular graphs and there better results could be 
possible.

%An analogous class of polytopes which we will be using is called neighborly 
%cubical polytopes. 
%
%\begin{definition}
% A neighborly cubical polytope $\NC_d(m)$ is a cubical $d$-polytope with the 
%$\left\lfloor \frac{d}{2} 
%\right\rfloor$-skeleton of a $m$-cube.
%\end{definition}
%
%They were first discovered by Joswig and Ziegler. Sanyal and Ziegler \cite{Z102}
%generalized the construction to show that each $(d-2)$-dimensional neighborly 
%polytope on $(m-1)$ vertices can be used to define a 
%different combinatorial type of $\NC_d(m)$. 
%
%\begin{conjecture}
% (Kalai 1991 \cite{Ka2}, corrected version 2004 \cite{kalai04:_polyt}) The 
%graph of a simple $d$-polytope can be separated by removing 
%$O(n^{(d-2)/(d-1)})$ vertices. 
%\end{conjecture}
%
%This reduces to the planar separator theorem in the 3-dimensional case. 



\chapter{A simple construction of simple polytopes}

In this chapter we construct polytopes which disprove conjecture \ref{simpleconjecture} and
calculate a lower bound on the size of the minimal separators in two ways. Both proofs are for 
4-dimensional case, but they easily generalize to higher dimensions. The first way requires
no advanced techniques, but the bound achieved is not sharp. For the second proof,
we restrict ourselves to a subset of the neighborly cubical polytopes where we have complete
characterization of the combinatorics. This combined with a lemma proven by Sinclair to study
the mixing properties of Markov chains gets us the sharp bound. The main result of this
thesis is the following theorem.


\begin{theorem}
\label{maintheorem}
 Let $d\geq 4$. There exist $d$-dimensional simple polytopes with
separators of size $\Omega(n/\log n)$.
\end{theorem}

\subsection{Neighborly cubical polytopes}
\ref{ncp}

\begin{definition}
A neighborly cubical polytope $\NC_d(m)$ is a cubical $d$-polytope with the $\left\lfloor \frac{d}{2} 
\right\rfloor$-skeleton of a $m$-cube.
\end{definition}

They were first discovered by Joswig and Ziegler \cite{Z62}. Sanyal and Ziegler \cite{Z102}
generalized the construction to show that each $(d-2)$-dimensional neighborly 
polytope on $(m-1)$ vertices can be used to define a 
different combinatorial type of $\NC_d(m)$. As the boundary complex of any $\NC_d(m)$
is a subcomplex of the boundary complex of the $m$-cube, its faces correspond to vectors
in $\{0,1,*\}^m$, or equivalently, $\{-,+,*\}^m$.


Our construction starts from the neighborly cubical 4-polytopes $\NC_4(m)$. 
Their graph is that of an $m$-cube. The separators of $m$-cubes were discovered by Harper 
\cite{Harp}.
While hypercubes are simple polytopes, $\NC_4(m)$ are not since each vertex is 
incident to $m>4$ edges. However, we can construct simple polytopes from them by 
truncating the edges and vertices.

 

\begin{lemma}
(See also Ewald \& Shephard \cite{EwSh}) A $d$-polytope can be made simple 
by $(d-2)$ iterative cuts.
\end{lemma}

\begin{proof}
This is easier to see in the dual case. The dual operation of cutting a vertex 
is stacking on a facet. By stacking on all facets one after the other, we 
remove all the original facets and introduce new ones, which are pyramids over 
$(d-2)$-faces. The original $(d-2)$-faces are preserved and we can stack on 
them, removing them and the facets created in the previous operation. Every 
iteration of this operation removes all the facets created in previous 
iterations and creates new facets which are iterated pyramids on successively 
lower dimensional faces. Once we have done it $(d-2)$ times, the iterated 
pyramids are over edges and hence simplices. This means that the facets are 
simplices, so the polytope is simplicial and its dual is simple.
\end{proof}

Therefore, by 2 iterations of cuts, we can turn $\NC_4(m)$ into a simple 
polytope. Let us see what this polytope looks like. 

The starting polytope $\NC_4(m)$ has the flag vector 
\begin{align*}
 flag(\NC_4(m)) &:= (f_0, f_1, f_2, f_3; f_{03}) \\
	&= (2^m, m2^{m-1}, 3(m-2)2^{m-2}, (m-2)2^{m-2}; 8(m-2)2^{m-2}) \\
	&= (4, 2m, 3(m-2), m-2; 8(m-2))2^{m-2}
\end{align*}

The first two entries follow from the fact that the graph is the same as that 
of the $m$-cube. The final entry follows from the fact that the facets are 
3-cubes and hence consists of 8 vertices. Since the facets are 3-cubes, they 
consists of 6 squares, and each of these squares is an intersection of two 
3-cubes. Hence we know that $3f_2 = f_3$ and we can calculate these values with 
the Euler-Poincare formula.

We denote the polytope formed by truncating all the vertices by $\NC_4(m)'$.
Truncating the vertices means removing all of the original vertices, but 
adding 2 for each edge, one at each end of the edge. The facets of this new 
polytope come in two types:

\begin{itemize}
 \item The $(m-2)2^{m-2}$ original facets, which are cubes whose vertices are 
cut off. These are simple polytopes with $f$-vector (24,36,14).
 \item The $2^m$ new facets which are the vertex figures. The facets of these 
facets are vertex figures of the cube and hence triangles. Therefore these 
new facets are simplicial. Their $f$-vector is ($m$, 
$3m-6, 2m-4)$.
\end{itemize}

This means that the flag vector of $\NC_4(m)'$ is 
\begin{equation}
 flag(\NC'_4(m)) = (4m, 14m-24, 11m-22, m+2; 28m-24)2^{m-2}
\end{equation}

By cutting of all of the original (but shortened) edges we arrive at 
$\NC_4(m)''$. It has three types of facets:

\begin{itemize}
 \item $(m-2)2^{m-2}$ simple polytopes which correspond to the facets of 
$\NC_4(m)$, but whose vertices and edges have been cut. Their $f$-vector is 
(48, 72, 26).
\item $m2^{m-1}$ prisms over polygons with $3$ to $m-1$ sides. They come from 
the edges of the original polytope.
\item $2^m$ simple polytopes which are the truncated vertex figures of 
$\NC_4(m)$. Their $f$-vector is $(6m - 12, 9m-18, 3m-4)$.
\end{itemize}

Using the knowledge of the facets we can work out the flag vector of 
$\NC_4(m)''$.

\begin{equation}
 flag(\NC_4(m)'') = (24m-48, 48m-96, 27m - 46, 3m+2; 28m -48)2^{m-2}
\end{equation}

Since $2f_0 = f_1$, we can verify that $\NC_4(m)''$ is indeed a simple 
polytope.

\section{Lower bound, straightforward approach}

In this section we shall show an easy proof for a bound on the separator. It
is enough to disprove \ref{simpleconjecture}, but not enough to verify Thurston's
claim.

We are interested in what the graph $G(\NC_4(m)'') = G''_m$ looks like. Its 
most striking feature is its similarity to the graph of the hypercube, $Q_m$. 
$Q_m$ is a graph on $2^m$ vertices with $m2^{m-1}$ edges. $G''_m$ 
similarly consists of $2^m$ groups of $(6m-12)$ vertices which we call 
\textit{clusters}. Each cluster is a 3-regular graph and the clusters are 
connected to each other by 3 to $m-1$ edges. We will use this similarity to 
prove a bound on the size of separators of $G''_m$. Let us therefore take a 
look at the separators of $Q_m$.

The problem of finding the sets of size $k$ with minimal size neighborhood is 
called the \textit{discrete isoperimetric problem}. It is a generalization of 
the minimum separator problem as there is no restriction on the size of $k$. 
Harper \cite{Harp} solved the problem on the hypercube in the 1960's. The 
answer turns out to be Hamming balls.

\begin{definition}
 (Hamming ball) Let $a,b \in \{0,1\}^n$. The Hamming distance between $a$ and 
$b$ is the number of elements where $a$ and $b$ differ. The Hamming ball 
$B(x,r)$ is the set of elements at distance $\leq r$ from $x$.
\end{definition}

A Hamming ball $B(x,r) \subset Q_m$ has 
\begin{equation}
 \sum_{i=0}^r {m \choose i}
\end{equation}
vertices. If $k$ can't be expressed in this form, let $r$ be maximal so that 
the sum is $\leq k$ and include some vertices with distance $r+1$ to the set. 
However, the choice is not arbitrary. Suppose $x$ is the vertex labeled with 
all zeros. Then the extra vertices are those which come first in 
\textit{lexicographic order} of vertices with $r+1$ ones and the rest zeros.
\begin{definition}
 (Lexicographic order) Let $a,b \in \{0,1\}^m$. Then $a<b$ if the first $i$ 
elements of $a$ and $b$ are the same but $a_{i+1} < b_{i+1}$.
\end{definition}
The sets with smallest vertex boundary hence form an increasing sequence i.e.\ 
the previous set is a subset of the following set. The order in which the 
vertices are added as $k$ increases is also called \textit{graded lexicographic 
order}.

Asymptotically, if $k = c2^m$ for some $0<c<1$ as $m\rightarrow \infty$, this 
means that the layer $r$ needs to be near the middle, i.e.\ $r\approx m/2$. 
%By Stirling's formula,

%\begin{equation}
% {m \choose m/2} = \Theta \left(\frac{2^m}{\sqrt{m}}\right)
%\end{equation}

Let $f: \NC_4(m)'' \rightarrow Q_m$ 
be the map that sends all the vertices of a cluster to the corresponding vertex 
in the cube. 
Let $(A,B,C)$ be a separator in $C''_m$. Then we can partition
$Q_m$ to sets $(a,b,c)$ using $f$. If $f^{-1}(v) \subset A$ (resp.) $B$ 
then 
$v \in a$ (resp.) $b$. Otherwise, $v \in c$. Since the clusters are connected 
graphs, any cluster which contains vertices in $A$ and $B$ must also contain 
verices in $C$. This means that $|c| \leq |C|$ as $f^{-1}(c) \cap C \neq 
\emptyset$.
Now, no vertices with labels $a$ and $b$ are adjacent since this would imply 
that a cluster which belongs to $A$ is adjacent to a cluster which belongs to 
$B$ which impossible in a separator. Therefore
\begin{enumerate}
 \item $(a,b,c)$ is a separator of $Q_m$ or
\item one or both of $a,b$ is small.
\end{enumerate}
In the first case, $c$ contains at least the neighborhood of $a$ and $b$ and 
hence its size is bounded by the isoperimetric inequality. Therefore $c$ has 
$\Omega(2^m/\sqrt{m})$ vertices. Hence $C$ contains at least $\Omega(N/(\log 
N)^{\frac{3}{2}}$.

In the latter case, $c$ must contain a linear number of vertices. This means 
that also the set $C$ must be large.  

\section{Sharp lower bound}

To prove the sharp bound, we need some extra tools. Let $G= (V,E)$ be a simple graph and
 $S\subset V$. The \textit{edge boundary} $\delta(S)$ is defined as the set of edges with one end
in $S$ and the other in $V\setminus S$. The \textit{edge expansion} $\mathcal{X}(G)$ is then defined
as 
\begin{equation}
\mathcal{X}(G) \coloneqq  \min \left\{ \frac{ |\delta(S) |}{ |S |}, S \subset V, S\neq \emptyset, |S| \leq \frac{ |V |}{2}  \right\}.
\end{equation} 
This quantity gives information on mixing properties of Markov chain on $G$. We study it since
a large edge expansion on a regular graph also implies a large separator (see \ref{separatorsize} below).
The edge expansion can be estimated by "canonical paths" method, invented by Sinclair \cite{Sinclair}. 
A good introduction, with applications to the graphs of polytopes, can be found in \cite{Kaibel}. 

\subsection{Sinclar's canonical paths}

To estimate edge expansion of a graph $G$, we define a path between every (ordered) pair of vertices $(s,t) \in V \times V$.
The intuition behind the method is that in graphs with small edge expansion, there exist edges which are "congested", that is there
are many paths going through them. To make this intuition precise, let $\phi: E(G) \rightarrow \mathbb{N}$ count the number of paths
that traverse this edge. The edge expansion can be bounded 
in terms of
\begin{equation}
\phi_{\max} \coloneqq \max \{ \phi(e), e \in E(G) \}.
\end{equation}


\begin{lemma}[Sinclair's lemma]
\label{Sinclair}
Let $\phi_{\max}$ be as defined in previous paragraph. Then the edge expansion of $G$ satisfies 
\begin{equation}
\mathcal{X}(G) \ge \frac{n}{2 \phi_{\max}}.
\end{equation}
\end{lemma}

\begin{proof}
Let $\phi(\delta(S))$ be the sum of $\phi(e)$ for all the edges in $\delta(S)$. Every path that has one end in $S$
and the other in $V\setminus S$ needs to use one of these edges, so we have $\phi(\delta(S)) \ge   |V\setminus S | |S |$.
On the other hands, $\phi_{\max}$ is the globabl maximum for all edges of the graph, so we have $\phi(\delta(S)) \le  |\delta(S) |\phi_{\max}$.
Hence for any $S \le \frac{ |V |}{2}$, we have
\begin{equation}
\label{sepbound}
 \mathcal{X}(G) \ge \frac{|\delta(S)|}{|S|} 
                \ge \frac{\phi(\delta(S))}{\phi_{\max}\,|S|}
                \ge \frac{|S|\,|V{\setminus}S|}{\phi_{\max}\,|S|}
                 =  \frac{|V{\setminus}S|}{\phi_{\max}}
				\ge \frac{n}{2\,\phi_{\max}}.
\end{equation}
\end{proof} 
\subsection{Relating edge expansion and separator size}

\begin{lemma}
\label{separatorsize}
Let $G$ be a $d$-regular on $n$ vertices with edge expansion $\mathcal{X}(G)$.
Then its separators are of size at least
\begin{equation}
	\frac{c}{d}\mathcal{X}(G)n = \Omega(\mathcal{X}(G)n),
\end{equation}
where $c$ is the constant from the definition of a separator.
\end{lemma}
\begin{proof}
Let $G$ be as defined above and let $(A,B,C)$ be a separator of $G$ with $|B| \ge |A| \ge cn$.
As there are no edges between $A$ and $B$, any edge in the set $\delta(A)$ has its other end in
$C$. This means that $\delta(A)$ contains at least $|A|\mathcal{X}(G)$ edges. Since $G$ is $d$-regular,
the size of $C$ is at least $\mathcal{X}(G)|A|/d \ge \mathcal{X}(G)cn/d = (c/d)\mathcal{X}(G)n$.  
\end{proof}

For the sharp bound, we will restrict to \textit{cyclic neighborly cubical polytopes} 
 $\NC^c_d(m)$.
This was the combinatorial type discovered by Joswig \& Ziegler \cite{Z62}. They
are also a result of using the construction method of Sanyal \& Ziegler \cite{Z102}
on cyclic polytopes with vertices in standard order, which justifies the name.  According
to their analysis, the vertex figures are combinatorially equivalent to a pyramid over
a triangulation of the cyclic polytope $C_{d-2}(m-1)$. Since the graph $G_m$ of $\NC^c_d(m)$
is isomorphic to that of the $m$-cube, we will label the vertices by vectors in $\{-,+\}^m$ and
edges by vectors in $\{-,+,*\}^m$ with exactly one $*$-entry. The index of the $*$-entry is
called the \textit{direction} of the edge. There are $2^{m-1}$ edges in each direction.
The combinatorics of $\NC^c_d(m)$ are fully described by Cubical Gale Evennes Criterion.
This criterion is fairly complicated even in the 4-dimensional case, but we do not need the 
complete description.

\begin{theorem}
[Part of the Cubical Gale Evenness Criterion, for $d=4$ {\cite[Thm.~18]{Z62}}] 
The facets of the cyclic neighborly polytope $\NCC_4(m)$ are given by vectors in
$\{-,+,*\}^m$ with exactly three $*$-entries. 

If the first component of a vector in $\{-,+,*\}^m$ is $*$, it corresponds to a facet 
of $\NCC_4(m)$ if and only if the rest of the vector satisfies the Gale Evenness criterion, 
that is, if between any two non-$*$-entries there is an even number of $*$-entries.
Equivalently, this happens if in the rest of that vector the two $*$-entries are
cyclically adjacent.
\end{theorem}

 This shows that for any vertex $v$, the edges in directions $i$ and $i+1$ span a 2-face, because
the vector $(*,v_2, \dots, v_{i-1}, *, *, v_{i+2}, \dots, v_m)$ corresponds to a facet of $\NCC(4)$
for $i\le 2 < m$. As the 2-faces correspond to edges in the vertex figure, this means that the vertices 
of the vertex figure at $v$ are cyclically connected in a consistent way indepdent of $v$. In other words
there is a Hamiltonian cycle $(123\dots m)$ going through the vertices in the graph of each figure.
This is also shows that the boundary complex of $\NCC(m)$ contains a copy of the polyhedral surface 
describe by Ringel \cite{ringel55:_ueber_probl_wuerf_wuerf} in 1957. This was also pointed out by
Ziegler \cite[Sect 3.]{Z100}. See also Joswig \& Rörig \cite{joswig:_neigh}.

Let us take a closer look at $\NCC_4(m)'$ first.  Each vertex in a "new" facet is labelled naturally by
the direction of the edge of $G_m$ it is incident to. Therefore we can label the vertices of $G'_m$
by $(v,i) \in \{-,+\}^m \times [m]$ where $v \in \{-,+\}^m$ is the label of the vertex of $\NCC_4(m)$ which has been truncated and $i \in [m]$ 
is the direction of the edge of $\NCC_4(m)$ which has been cut. The vertex figures of $G'_5$ are illustrated in \ref{figure:graphs}.

We divide the edges of $\NCC_4(m)'$ into three different classes. First of all, there are the "long" edges which correspond to the edges $\NCC_4(m)$.
They are incident to vertices $(v,i)$ and $(v',i)$ where $v$ only differ in coordinate $i$. There $m2^{m-1}$ such edges in total. Secondly, 
there are "medium" edges, which are between vertices $(v,i)$ and $(v,i+1\mod m)$. They form a Hamilton cycle $(1\,2\,3\,\dots\, m)$ on each cluster $v$. In total there are $m2^m$ medium edges. The rest of the edges are "extra", they are between vertices $(v,i), (v,j)$ with $j-i \neq \pm 1 \mod m$.
There are $(2m-6)2^m$ "extra" edges in total. The "extra" and "medium" edges together form the vertex figures of $\NCC_4(m)$ which we call clusters. 

Truncating the long edges of $\NCC_4(m)'$ gets us the polytope $\NCC_4(m)''$. The "medium" and "extra" edges are shortened by this procedure, but are otherwise unchanged. The "long" edges are replaced by their edge figures, which are prisms over $k$-gons with $3 \le k \le m-1$. Here $k$ is the degree of the incident vertices in their clusters, or equivalently their degree minus 1. We call the edges between these $k$-gons a \textit{parallel class of long edges}. The $k$-gon consist of "short" edges. A picture of a cluster of $G''_m$ can be seen in the bottom figure of \ref{figure:graphs}.

The cycles of short edges naturally correspond to vertices of $G_m'$, so we extend the notation $\{-,+\}^m \times [m]$ to them. "Medium" edges connect subsequent cycles $(v,i), (v,i+1 \mod m)$ while "extra" edges connect non-subsequent cycles. 

\subsection{Canonical paths for $G''_m$}
\label{canpaths}

Now we will construct the canonical paths required for \ref{Sinclair} in $G''_m$. Our construction relies on the similarity
of the vertex figures and the cubical connections between the clusters. 

Let $v_0$ be a vertex in the cycle of short edges $(v,i)$ and $w_0$ be a vertex in the cycle of short edges $(w,j)$. 
We will construct a path from $v_0$ to $w_0$ in the following way. Consider the coordinates in cyclic order $i, i+1, \dots m, 1, \dots,
i-1$ and for each coordinate do the following:

\begin{quote}
\textbf{Procedure P:}
Suppose the path we have constructed so far ends at vertex $u_0 \in (u,k)$.
\begin{compactitem}
\item If $u$ and $w$ differ in coordinate $k$, take the long edge from $u_0$ to $(u',k)$,
where $u$ and $u'$ differ only in coordinate $k$. Then, using only short edges and a medium edge,
move to the cycle of short edges labelled $(u', k+1)$.
\item Otherwise, $u$ and $w$ agree in coordinate $k$. Using only short edges and a medium edge,
move to the cycle $(u, k+1)$.
\end{compactitem}
\end{quote}

After performing this procedure at most $m$ times, we end up in the cluster $w$. After at most $m-1$ more
iterations, we end up in the cycle $(w,j)$. In the final iteration, we take at most $m-1$ steps in this cycle to arrive
at $w_0$.

We will now show that no edges is used more than $(O(m^2 2^,)$ times by considering the four cases of short, medium, long
and extra edges separately.

\afterpage{% 
\begin{figure}
	\input figure1.tex
\bigskip

	\input figure2.tex
\bigskip

	\input figure3.tex 
	
\caption{The graphs $\CG_m$, $\CG_m'$, and $\CG_m''$, for $m=5$:
	Local situation at one cluster.}
\label{figure:graphs}
\end{figure}	
\clearpage
}

\subsection{Long edges}

On average, a random pair of clusters differs in $\frac{m}{2}$ coordinates. Therefore half of the $n^2$ paths use long edges in direction $i$.
Our definition of paths is independent on the value of the other coordinates. so each of the $2^{m-1}$ parallel classes in direction $i$ are used
equally often. Every parallel class, and therefore every edge, is used by $\frac{n}{2} / 2^{m-1}$ paths.  The number of vertices is $n \in O(m2^m)$, so this works out to be $O(m^2 2^m)$.

\subsection{Medium and short edges}

Every path we construct uses at most $2m$ iterations of procedure $P$. The total number of iterations over all paths is therefore at most
$O(2mn^2)$.  Every iteration uses at most one long edge, one short edge and short edges
on one cycle. The extra edges are not used at all. 

Our construction treats all coordinates and both values of the coordinate equally, each set of "cycle of short edgess and a medium edge leaving it"
are used equally often. Moreover, these sets are disjoint. Therefore every short and medium edge is used at most $2mn^2 / (m2^m)$ times. 
As $n \in O(m^2 2^m)$, this is works out to be $O(2mn^2 / m2^m) = O(m^2 2^m)$. 

\subsection{Wrapping things up}

In section \ref{ncp} we showed that $G''_m$ has $(6m-12)2^m$ vertices. Let us
now combine our canonical paths with Sinclair's lemma \ref{Sinclair}:
\begin{equation}
\mathcal{X}(\CG_m'') 
\ge \frac{n}{2\,\phi_{\max}} 
\ge \frac{n}{2n^2/2^m}
  = \frac{2^m}{2n}
  = \frac{2^m}{2(6m-12)2^m}
  = \frac{1}{12(m-2)}. 
\end{equation}
With lemma \ref{sepbound}, we get a bound on the size of minimal separators:
\begin{equation}
\frac c4 \mathcal{X}(\CG_m'')n 
\ge \frac{c\,n}{48(m-2)} 
 =  \Omega\Big(\frac n{\log n}\Big),
\end{equation}
as $n = (6m-12)2^m$ and thus $m=\Theta(\log n)$.

\section{Upper bound on minimal separators}

The minimal separators of $C''_m$ have $O(n/\log n)$ vertices. By setting $A$ 
and $B$ to two $(m-1)$-cubes along a random ``edge direction'' of $Q_m$ we get 
two sets with $3\cdot 2^m$ edges going between them. Moving either end vertex 
of the edge to set $C$ we get a separator with $3\cdot 2^m = O(n/\log(n))$ 
edges.

\section{Higher dimensions}

This construction generalizes easily to higher dimensions. There are two 
natural ways, using $\NC_d(m)$ i.e.\ higher dimensional neighborly polytope, or 
taking prisms over $\NC_4(m)''$.  In the first case we have to do more cuts to 
make the polytope simple, resulting in a large number of vertices. Hence the 
second way is the correct one. 

Let $G$ be the graph of a $(d-4)$-fold prism over $\NC_4(m)''$ for $d\geq 4$. 
Any separator 
$(A,B,C)$ of $\NC_4(m)''$ can be turned into a separator of $G$ by setting 
$(A',B',C')$ to be the prisms over their respective sets. Since no vertices 
$a\in A, b\in B$ are adjacent and all the copies of a single vertex are in the 
same set, there are no edges between vertices in $A'$ and $B'$. This means that 
$G$ is at least as easy to separate as $\NC''_4(m)$. On the other hand, $G$ 
also has a cubelike structure and hence with the discrete isoperimetric 
inequality we get the same upper bound as for $\NC''_4(m)$. 




