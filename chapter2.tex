\chapter{Expander graphs}
\label{mathchapter}


The main objects of study in this work are polytopes. They are interesting 
objects in their own right but also play an important role in e.g. 
optimization. In the following section we introduces the basics of polytopes. 
More in-depth knowledge can be found in Ziegler's book [citation]. There are 
two ways to define a polytope, the vertex (or interior) representation and the 
half-space (or exterior) representation.

\begin{definition}
(V-Polytope) A polytope is the convex hull of a finite number of points in 
$\mathbb{R}^d$, that is 
\begin{equation}
 P = \left\{x\in \mathbb{R}^d \middle| x = \sum_{i=1}^{|V|} \lambda_i v_i, 
0\leq \lambda_i \leq 1, \sum_{i=1}^{|V|} \lambda_i = 1 \right\}
\end{equation}

\end{definition}

\begin{definition}
 (H-polytope) A polytope is the bounded intersection of a finite number of 
half-space, that is
\begin{equation}
 P = \bigcap_{i=1}^n \{x \in \mathbb{R}^d \mid \mathbf{a}_i^T \mathbf{x} \leq 
c_i \}
\end{equation}
where $a_i \in \mathbb{R}^d$ are normal vectors of the hyperplanes and $c_i \in 
\mathbb{R}$ are displacements. If this description is minimal with respect to 
the number of half-spaces required to describe $P$, it is called irredundant.
\end{definition}

It is a fundamental theorem of polytopes that these two representations are 
equivalent.

The dimension of a polytope is the dimension of its affine hull. The affine 
hull of a set of points is the smallest hyperplane which contains all of the 
points. If the lowest dimensional hyperplane which contains $n$ points is 
$(n-1)$-dimensional, the points are set to be affinely independent. The 
boundary 
of a polytope consists of lower dimensional polytopes called faces. 
\begin{definition}
 (Face) A face is an intersection of a polytope with a hyperplane 
\begin{equation}
 F = P \cap \{x \in \mathbb{R}^d \mid \mathbf{a}\mathbf{x} = c\}
\end{equation}
where the hyperplane has to be \textit{valid}, that is $\mathbf{a}\mathbf{x} 
\leq c$ holds for all points in $P$. 

\end{definition}

Some faces have special names: the 0-dimensional faces are called 
\textit{vertices}, the 1-dimensional faces are \textit{edges} and the 
$d-1$-dimensional faces are \textit{facets}.
 The faces of a polytope 
form a lattice with ordering given by inclusion. This lattice is called the 
\textit{face lattice} of the polytope.

The faces of various dimensions 
are counted by the face vector.
\begin{definition}
 (Face vector) The face vector is $(f_0, f_1, \dots, f_{d-1})$ where $f_i$ is 
the number of $i$-dimensional faces.
\end{definition}
The face vector satisfies the famous Euler-Poincare equation
\begin{equation}
 \sum_{i=0}^{d-1} (-1)^i f_i = 1 - (-1)^d.
\end{equation}
This is the only linear equation satisfied by the face vector. However, we can 
find more constraints by studying the \textit{flag vector}.

\begin{definition}
 (Flag vector) The flag vector counts the chains of faces in specified 
dimensions.
\end{definition}

For example, the entry $f_{013}$ counts the number of chains in the face 
lattice which has entries in dimensions $(0,1,3)$. However, many of the entries 
in the flag vector are redundant.

A polytopal complex is a collection of polytopes that fit together nicely.
More precisely:
\begin{definition}
 (Polytopal complex) A polytopal complex $\mathfrak{C}$ is a finite set of 
polytopes with the following properties:
\begin{enumerate}
 \item The empty polytope $\emptyset \in \mathfrak{C}$.
  \item (Downward closure) If $F \subset P$ is a face of $P$ and $P \in 
\mathfrak{C}$, then 
$F \in \mathfrak{C}$.
\item The intersection $P\cap Q$ of $P,Q \in \mathfrak{C}$ is a face of both.

\end{enumerate}

\end{definition}

% Examples of polytopal complexes include the polytope itself and its 
% \textit{boundary complex} which is the collection of all proper faces of $P$. 
Any polytope $P$ is a polytopal complex, as is the \textit{boundary complex} of 
$P$. The boundary complex is the collection of proper faces of $P$.
The $k$-skeleton of a polytope is the collection of all faces of dimension $\leq 
k$. Hence the $d$-skeleton of a $d$-polytope is the polytope itself and the 
boundary complex is the $(d-1)$-skeleton. The 1-skeleton of a polytope is called 
its graph and will be important for us later on.

% Each polytope has a dual polytope associated to it. Its face lattice is the 
% lattice of the original polytope ``upside down''. Hence its $i$-faces 
% correspond to the $(d-i)$-dimensional faces of the original. The dual of the 
% dual is the original polytope. 

Two polytopes are said to be dual to each other if there is an inclusion 
reversing bijection between their face lattices. Clearly, the dual of a 
$d$-polytope is another $d$-polytope. The fact that every polytope has a dual 
can be seen from the \textit{polar set}. Let $A\subset \mathbb{R}^d$, then the 
polar set $A^*$ is defined by
\begin{equation}
 A^* = \{y \in \mathbb{R}^d \mid \langle x,y\rangle \leq 1 \, \forall  x  \in A 
\} 
\end{equation}
If $A$ is a convex compact set with $0 \in$ int $A$, so is $A^*$. In 
particular, if $A$ is a polytope containing $0$ in the interior, $A^*$ is dual 
to it. 

In any dimension, the polytope with the fewest vertices is the simplex. 
\begin{definition}
 (Simplex) A $d$-simplex is the convex hull of $d+1$ affinely independant 
points.
\end{definition}
The faces of a simplex are lower dimensional simplices. A polytope whose facets 
(but not necessarily the polytope itself) are simplices is called 
\textit{simplicial}. The dual of a simplicial polytope is called 
\textit{simple}. The vertices of a simple $d$-polytope are contained in exactly 
$d$ edges and $d$ facets. 

Another important polytope is the $n$-cube.

\begin{definition}
 The $d$-cube (or hypercube) is a defined by the inequalities
\begin{equation}
 0 \leq x_i \leq 1 
\end{equation}
for $i \in 1,\dots, n$.
\end{definition}
The $d$-cube has $2^d$ vertices, $n2^{d-1}$ edges and $2d$ facets which
are $(d-1)$-cubes.

Two polytopes $P\subset \mathbb{R}^d, Q \subset \mathbb{R}^e$ are 
\textit{affinely isomorphic} if there is an affine map $f: \mathbb{R}^d 
\rightarrow \mathbb{R}^e$ which is a bijection between the points of the 
polytope. A more relaxed equivalence relation is combinatorial equivalence. Two 
polytopes are combinatorially equivalent if their face lattices are isomorphic 
i.e. there is an bijection between the elements of the lattice which respects 
the inclusion relation. 

Cutting of a vertex is a way to get a new polytope from an old one. This is 
done by adding a new inequality to the description. This inequality is not 
valid for the vertex which is cut off but is valid for all other vertices. All 
the polytopes formed this way are combinatorially equivalent (but not 
necessarily affinely isomorphic). For each face intersected by the hyperplane, 
a face of one dimension lower is added to the face lattice. The new facet is 
called the \textit{vertex figure}. Vertex cuts can be generalized to cutting of 
faces of higher dimensions by requiring the added inequality to be violated by 
all the vertices in the face. The dual operation is called stacking on a facet. 
It means adding a new vertex to the polytope which satisfies all the 
inequalities in the irredundant description with strict inequality except one 
(which is violated).

Graphs appear everywhere in mathematics but they are especially common in 
combinatorics. 

\begin{definition}
(Graph) A graph is an ordered pair $G=(V,E)$. The elements of $V$ are called 
vertices. The elements of $E$ are 2-element subsets of $V$ and they are called 
the edges of the graph. 
\end{definition}

If $(u,v) \in E$, the vertices $u$ and $v$ are said to be \textit{adjacent}.

A $d$-regular graph is a graph where every vertex is incident to a $d$ edges.

The connectivity of graph is the minimum number of vertices or edges that are 
required to be removed in order to disconnect the graph The graphs of 
3-polytopes are precisely all planar, simple, 3-connected graphs. However, this 
often means that one of the components is very small. A more nuanced view is 
provided 
by separators:

\begin{definition}
(Separator) A (vertex) separator is a partition of the vertices of a graph into 
sets $(A,B,C)$ where $|A|, |B| \geq c|V|$ where $0<c<1/2$ is called the 
separation 
constant. The size of the separator is $|C|$.  
\end{definition}

Another way to think about separators is by minimizing \textit{neighborhoods} 
of a set of given size.

\begin{definition}
 (Neighborhood) Let $S \subset V$ be a set of vertices. Then its neighborhood 
$N(S)$ is the set $\{v \in V\setminus S \mid (u,v) \in E, u \in S\}$. 
\end{definition}


Separators can also be defined by removing edges instead of vertices. However, 
for regular graphs the sizes of these separators differ by a constant multiple 
so we ignore the distinction.

The size of minimal separators in the graphs of 3-polytopes was discovered by 
Lipton and Tarjan [refhere] in 1979. They showed that planar graphs have 
separators of size $O(\sqrt{n})$ where $n$ is the number of vertices. This is 
the best possible result since an $mxm$-grid requires $m$ vertices to be 
removed to be separated. However, in higher dimensions such results are 
impossible due to the existance of \textit{neighborly polytopes}.

\begin{definition}
 A neighborly polytope on $n$-vertices is a $d$-polytope with the $\left\lfloor 
\frac{d}{2} \right\rfloor$-skeleton of a $n$-simplex.
\end{definition}

Hence any neighborly polytope of dimension $\geq 4$ has the complete graph as 
its graph and can't be separated by removing vertices. However, simple 
$d$-polytopes have $d$-regular graphs and there better results could be 
possible. 

An analogous class of polytopes which we will be using is called neighbourly 
cubical polytopes. 

\begin{definition}
 A neighborly cubical polytope $\NC_d(m)$ is a $d$-polytope with the 
$\left\lfloor \frac{d}{2} 
\right\rfloor$-skeleton of a $m$-cube and whose facets are $d-1$ cubes.
\end{definition}

They were first discovered by Joswig and Ziegler.

\begin{conjecture}
 (Kalai 1991/2004) The graph of a simple $d$-polytope can be separated by 
removing $O(n^{(d-2)/(d-1)})$ vertices. 
\end{conjecture}

This reduces to the planar separator theorem in the 3-dimensional case. 

\begin{theorem}
 There exist $d$-dimensional simple polytopes which require separators of size 
$\Theta(n/(\log n)^{\frac{3}{2}})$.
\end{theorem}

Our construction starts from the neighborly cubical 4-polytopes $\NC_4(m)$. 
While hypercubes are simple polytopesm, $\NC_4(m)$ are not since each vertex is 
incident to $m>d$ edges. 

\begin{lemma}
 A $d$-polytope can be made simple by $(d-2)$ iterative cuts.
\end{lemma}

\begin{proof}
This is easier to see in the dual case. The dual operation of cutting a vertex 
is stacking on a facet. By stacking on all facets one after the other, we 
remove all the original facets and introduce new ones, which are pyramids over 
$(d-2)$-faces. The original $(d-2)$-faces are preserved and we can stack on 
them, removing them and the facets created in the previous operation. Every 
iteration of this operation removes all the facets created in previous 
iterations and creates new facets which are iterated pyramids on successively 
lower dimensional faces. Once we have done it $(d-2)$ times, the iterated 
pyramids are over edges and hence simplices.  
\end{proof}

Therefore, by 2 iterations of cuts, we can turn $\NC_4(m)$ into a simple 
polytope. Let us see what this polytope looks like. 

The starting polytope $\NC_4(m)$ has the flag vector 
\begin{align}
 flag(\NC_4(m)) &:= (f_0, f_1, f_2, f_3; f_{03}) \\
	&= (2^m, m2^{m-1}, 3(m-2)2^{m-2}, (m-2)2^{m-2}; 8(m-2)2^{m-2}) \\
	&= (4, 2m, 3(m-2), m-2; 8(m-2))2^{m-2}
\end{align}

The first two entries follow from the fact that the graph is the same as that 
of the $m$-cube. The final entry follows from the fact that the facets are 
cubes and hence consists of 8 vertices. Since the facets are cubes, they 
consists of 6 squares, and each of these squares is an intersection of two 
cubes. Hence we know that $3f_2 = f_3$ and we can calculate these values with 
the Euler-Poincare formula.

We denote the polytope formed by cutting of all the vertices by $\NC'_4(m)$.
Cutting off all the vertices means removing all of the original vertices, but 
adding 2 for each edge, one for each end of the edge. The facets of this new 
polytope come in two types:

\begin{itemize}
 \item The $(m-2)2^{m-2}$ original facets, which are cubes whose vertices are 
cut off. These are simple polytopes with $f$-vector (24,36,14).
 \item The $2^m$ new facets which are the vertex figures. The facets of these 
facets are vertex figures of the cube and hence triangles. Therefore these 
new facets are simplicial. Their $f$-vector is ($m$, 
$3m-6, 2m-4)$.
\end{itemize}

This means that the flag vector of $\NC'_4(m)$ is 
\begin{equation}
 flag(\NC'_4(m)) = (4m, 14m-24, 11m-22, m+2; 28m-24)2^{m-2})
\end{equation}

By cutting of all of the original (but shortened) edges we arrive at 
$\NC''_4(m)$. It has three types of facets:

\begin{itemize}
 \item $(m-2)2^{m-2}$ simple polytopes which correspond to the facets of 
$\NC_4(m)$, but whose vertices and edges have been cut. Their $f$-vector is 
(48, 72, 26).
\item $m2^{m-1}$ prisms over polygons with $3$ to $m-1$ sides. They come from 
the edges of the original polytope.
\item $2^m$ simple polytopes which are the truncated vertex figures of 
$\NC_4(m)$. Their $f$-vector is $(6m - 12, 9m-18, 3m-4)$.
\end{itemize}

Using the knowledge of the facets we can work out the flag vector of 
$\NC''_4(m)$.

\begin{equation}
 flag(\NC''_4(m)) = (24m-48, 48m-96, 27m - 46, 3m+2; 28m -48)2^{m-2}
\end{equation}

Since $2f_0 = f_1$, we can verify that $\NC''_4(m)$ is indeed a simple 
polytope.

\subsection{Horsey Porsey}

We are interested in what the graph $G(\NC''_4(m)$ looks like. Its most 
striking feature is its similarity to the graph of the hypercube, $C_m$. $C_m$ 
is a graph on $2^m$ vertices with $m2^{m-1}$ edges. $G(\NC''_4(m)$ similarly 
consists of $2^m$ groups of $(6m-12)$ vertices which we call \textit{clusters}. 
Each cluster is a 3-regular graph and the clusters are connected to each other 
by 3 to $m-1$ edges. 

Harper solved the \textit{discrete isoperimetric problem} on the hypercube. The 
problem asks 

\section{The End}
\label{sec:end}

Finally, this is the end.  The bibliography starts on
the next page.
Note how the \verb2\hyperref2 package
(mentioned in chapter \ref{introchap})
also makes hyperlinks from references
(e.g., Mulick\cite{mulick})
to entries in the bibliography.

