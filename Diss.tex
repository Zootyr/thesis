\documentclass[a4paper,12pt]{book}

%Packages
%\usepackage{amsfonts}
\usepackage{amsmath}
\usepackage{amssymb}
\usepackage{amsthm}
\usepackage{bm}
\usepackage{mathtools}
\usepackage[utf8]{inputenc}

\usepackage[pdftex]{graphicx}
\usepackage[dvipsnames]{xcolor}

\usepackage{afterpage}

\usepackage{paralist}
%\usepackage[colorlinks,citecolor=blue,pagebackref=true,pdftex]{hyperref}
\usepackage{hyperref}

\usepackage{tabularx}

\usepackage[ngerman,english]{babel}



\usepackage{caption}
\usepackage{subcaption}

\providecommand*\showkeyslabelformat[1]{\fbox{\tiny\tt#1}}
%\usepackage[notref,notcite]{showkeys}

\usepackage{paralist}
\usepackage{nicefrac}

\usepackage{tikz}
\usetikzlibrary{positioning}
\usetikzlibrary{arrows}

\usepackage{chngcntr}
\counterwithin{figure}{chapter}
\counterwithin{table}{chapter}

\usepackage[headsepline]{scrpage2}
\pagestyle{scrheadings}

%\usepackage[margin=3.5cm]{geometry}
\usepackage[left=3cm,right=4cm, bottom=3.5cm]{geometry}


%\setcounter{chapter}{-1}


%Layout
\newcommand\note[1]{\textbf{\color{blue}#1}}
\newcommand\Defn[1]{\textbf{#1}}
\newcommand\WORKLINE{\ \\ \rule{.42\textwidth}{3px} { WORKLINE }
\rule{.42\textwidth}{3px} \ \\}
    

%Operators
\DeclareMathOperator{\aff}{aff}
\DeclareMathOperator{\conv}{conv}
\DeclareMathOperator{\cone}{cone}
\DeclareMathOperator{\relint}{relint}
\DeclareMathOperator{\nvol}{Vol}
\DeclareMathOperator{\vol}{vol}
\DeclareMathOperator{\rk}{rk}
\DeclareMathOperator{\supp}{supp}
\DeclareMathOperator{\sgn}{sgn}
\DeclareMathOperator{\im}{im}
\DeclareMathOperator{\Crit}{Crit}

\usepackage{tikz}
%\usepackage{pgfmath}
\usepackage{xcolor}
\usetikzlibrary{shapes,shadows,arrows,patterns,trees,automata,positioning,snakes,fadings,backgrounds,intersections,calc}
\usepackage{tkz-euclide}
%\usepackage{pst-plot}
\usepackage{ifthen}
%%%%%%%%%%%% Definitions & styles %%%%%%%%%%%%%%%%%%%%
\tikzstyle{edge}=[line width=1mm, color=blue] %hamilton cycle edges
\tikzstyle{face}=[color=black,fill=white,opacity=0.8,line width=0.7mm]
\tikzstyle{line}=[black!50!white, opacity=1, line width=0.45mm] %long lines
\tikzstyle{label} = [color=black, font=\normalsize, opacity=1]
\tikzstyle{vertex} = [outer color=black,draw, color=black,line width=0.03mm, inner color=black!55!white, circle,inner sep=0.7mm,minimum size=0.9mm]



%Theorems
\theoremstyle{plain}
\newtheorem{theorem}{Theorem}[section]
\newtheorem{maintheorem}[theorem]{Main Theorem}
\newtheorem{lemma}[theorem]{Lemma}
\newtheorem{claim}[theorem]{Claim}
\newtheorem{corollary}[theorem]{Corollary}
\newtheorem{proposition}[theorem]{Proposition}
\newtheorem{conjecture}[theorem]{Conjecture}
\newtheorem{assumption}[theorem]{Assumption}
\newtheorem{convention}[theorem]{Convention}
\newtheorem*{theorem*}{Theorem}
\newtheorem*{mtheorem}{Main Theorem}
\newtheorem*{gtheorem}{General Theorem}

\theoremstyle{definition}
\newtheorem{definition}[theorem]{Definition}
\newtheorem{addendum}[theorem]{Addendum}
\newtheorem{example}[theorem]{Example}
\newtheorem{remark}[theorem]{Remark}
\newtheorem{notation}[theorem]{Notation}

\DeclarePairedDelimiter\floor{\lfloor}{\rfloor}
\newcommand\R{\mathbb{R}}
\newcommand\NC{\textrm{NCP}}
\newcommand\NCC{\mathrm{NC}^c}
\newcommand\CG{\mathrm{G}}


\begin{document}

 \pagenumbering{roman} 
\pagestyle{empty}

\begin{titlepage}
\setlength{\hoffset}{5mm}
\begin{center}
%\vspace*{1cm}
\begin{figure}[h]
\begin{minipage}{.35\textwidth}
\begin{center}
\includegraphics[width=3.5cm]{FU-logo.jpg}
\end{center}
\end{minipage}
\begin{minipage}{.64\textwidth}
\large {Fachbereich Mathematik und Informatik\\
der Freien Universit\"at Berlin}
\end{minipage}
\end{figure}

%\vspace{1cm}
\vspace{1.5cm}
\LARGE{\bf{
Separators of simple polytopes}}\\

\vspace{1cm}
\begin{figure}[h]
\begin{center}

%\includegraphics[width=6cm]{pictures/AP_subdiv.jpg}
\includegraphics[width=8cm]{Tesseract_Mark}

\end{center}
\end{figure}
\vspace{0.5cm}
\LARGE{
{Dissertation}\\}
\large
\vspace{1cm}
eingereicht von\\
Lauri Loiskekoski\\
\vspace{1.5cm}
Berlin 2017\\
\vspace{3cm}
\end{center}
\end{titlepage}

\cleardoublepage\thispagestyle{empty}

\begin{titlepage}
%\setlength{\hoffset}{5mm}
\begin{center}
%\vspace*{1cm}
\begin{figure}[h]
\begin{minipage}{.35\textwidth}
\begin{center}
\includegraphics[width=3.5cm]{FU-logo.jpg}
\end{center}
\end{minipage}
\begin{minipage}{.64\textwidth}
\large {Fachbereich Mathematik und Informatik\\
der Freien Universit\"at Berlin}
\end{minipage}
\end{figure}

%\vspace{1cm}
\vspace{1.5cm}
\LARGE{\bf{
Separators of simple polytopes}}\\

\vspace{1cm}
\begin{figure}[h]
\begin{center}

%\includegraphics[width=6cm]{pictures/AP_subdiv.jpg}
\includegraphics[width=8cm]{Tesseract_Mark}

\end{center}
\end{figure}
\vspace{0.5cm}
\LARGE{
{Dissertation}\\}
\large
\vspace{1cm}
eingereicht von\\
Lauri Loiskekoski\\
\vspace{1.5cm}
Berlin 2017\\
\vspace{3cm}
\end{center}
\end{titlepage}

\cleardoublepage\thispagestyle{empty}

\null\vfill
\begin{center}
\large{Advisor and first reviewer:\\
Prof. Dr. Günter M. Ziegler}\\
\vspace{0.2cm}
Second reviewer:\\
--\\
\vspace{0.2cm}
Third reviewer:\\
--\\
\vspace{0.6cm}
Date of the defense:\\
--
\end{center}

\cleardoublepage




\pagestyle{scrheadings}

\newcommand{\header}[1]{\markboth{}{\MakeUppercase{ #1}}}

%\phantomsection
%\addcontentsline{toc}{section}{Summary}

\phantomsection
\chapter*{Acknowledgements}
\header{Acknowledgements}
\addcontentsline{toc}{section}{Acknowledgements}

My deepest thanks go to my 




\cleardoublepage

\tableofcontents

\cleardoublepage

 \pagenumbering{arabic} 

\chapter{Introduction}
\label{intro}

Graphs are objects used to model pairwise relations between objects. For applications such as computer
networks, it is often important that the connections between the nodes are robust. This means that the failure of a few nodes doesn't 
break the network into disconnected parts. Separators are one way to model the connectedness of a graph. 

A separator is a partition of the vertices of a graph into three sets. There are two ``large" sets $A$ and $B$ with no edges
between a vertex in $A$ and a vertex in $B$, and 
an additional set $C$ containing the rest of the vertices. As moving vertices from $A$ or $B$ to
$C$ preserves the separation property, we are interested in the minimum size of the set $C$. 
The size of the set $C$ is called the size of the separator. A similar definition exists for edge separators, but we study
only regular graphs where their sizes are within a multiplicative constant (see lemma \ref{separatorsize}).
Separators are closely connected to the expander graphs; a family of expander graphs has minimal separators
whose size is linear in the number of vertices.

The story of separators of polytopes begins in 1951 with Ungar \cite{Ungar1951},
who proved that planar graphs, and therefore graphs of 3-polytopes, have minimal 
separators of size $O(\sqrt{n} \log n)$. Lipton and Tarjan proved the planar 
separator theorem in 1979 \cite{LiTa}. It gives the sharp bound
$O(\sqrt{n})$ provided that the large sets $A$ and $B$ both contain at least $\nicefrac{1}{3}$ of the vertices.
This result is best possible, since the $m\times m$ grid graph
can't be separated by a separator smaller than $\Omega(m)$. Miller and Thurston \cite{MillerThurston-separators} gave a 
geometric proof of the planar separation theorem using the  
Koebe–Andreev–Thurston circle-packing theorem.

Miller, Teng, Thurston and Vavasis \cite{MillerTengThurstonVavasis} generalized 
the theorem to intersection graphs of ball packings in $d$ dimensions. This 
led Kalai to conjecture in his 1991 paper \cite[Conj 12.1.]{Ka2}(repeated in the 1997 first edition of \cite{Kal97} ) that 
simple $d$-polytopes would have separators of size
\begin{equation}
O\left(n^{1-\frac{1}{\lfloor d/2 \rfloor}}\right),
\end{equation}
which fails for $d=3$ and for $d=4$ postulates separators of size $O(\sqrt{n})$. In the 2004 second edition of
the \textit{Handbook} \cite[Conj. 20.2.12]{kalai04:_polyt} he revised the conjecture to the following form:
\begin{conjecture}[Kalai]
\label{simpleconjecture}
Simple $d$-polytopes have separators of size
$O \left(n^{\frac{d-2}{d-1}}\right)$.
\end{conjecture}
In dimension 3, this gives the planar separator theorem, while in dimension 4 it postulates the existence
of separators of size $O(n^{2/3}).$

Simplicity is a necessary assumption as the graph of the cyclic polytopes is complete for 
$d\geq 4$ and hence has no vertex separator. Kalai also refers to 
\cite{MillerTengThurstonVavasis} for a claim by Thurston who stated that there are triangulations of 
$S^3$ with $n$ facets whose dual graphs have separators with size $\Omega(n/\log n)$. This is not 
stated in the paper, but refers to a construction which Thurston had described to his 
coauthors. Gary Miller told us (personal communication) that 
“\textit{Thurston gave an embedding of the cube-connected cycle graph in $\mathbb{R}^3$
as linear tets} [tetrahedra]”. The details of this construction appear to be lost.

Here we disprove conjecture \ref{simpleconjecture} and prove a stronger version of Thurston's claim: There is not only a triangulated simplicial sphere
on $n$ vertices, whose dual graphs has separators of size $\Omega(n/\log n)$, but even a simplicial convex 4-polytope
with this property. Our main theorem describes the simple dual polytope.

\begin{theorem}
There is a family of simple convex $4$-dimensional polytopes $\NCC_4(m)''$ with $n=\Theta(m 2^m)$ vertices for which
the smallest separators of the graph have size $\Theta(2^m) = \Theta(n/\log n)$.
\end{theorem}

This thesis is structured as follows. In chapter \ref{mathchapter} we define our main objects of study,
separators of graphs. This section also contains necessary background information on 
graphs and polytopes. In chapter \ref{ncpconstruction} we build the simple polytope used for our 
lower bound and calculate the size of their separators. Finally in chapter \ref{genes} we generalize
the construction to higher dimensions and discuss possible constructions for polytopes with even larger
minimal separators.

\chapter{Graphs and Polytopes}
\label{mathchapter}

\section{Polytopes}

The main objects of study in this work are polytopes and their graphs. Polytopes are 
interesting 
objects in their own right, but also play an important role in
optimization. In the following section we introduce the basics of polytopes. 
A more thorough treatment can be found in books by Gr\"unbaum \cite{Grunbaum69convexpolytopes} and Ziegler \cite{PolyLec} . 

Polytopes are geometrical objects, but we are mostly interested in their combinatorial properties.
We will introduce polytopes in their geometric representation, but quickly forget all about coordinates
and focus on things that can be worked out in the face lattice.

There are two ways to define a polytope, the vertex (or interior) representation and the 
half-space (or exterior) representation.

\begin{definition}
(V-Polytope) An \emph{$V$-polytope} is the convex hull of a finite set of points $V 
\subset \mathbb{R}^d$, that is 
\begin{equation}
 P = \left\{x\in \mathbb{R}^d \middle| x = \sum_{i=1}^{|V|} \lambda_i v_i, 
0\leq \lambda_i \leq 1, \sum_{i=1}^{|V|} \lambda_i = 1 \right\}.
\end{equation}

The inclusion-minimal set $V$ for which conv$(V) = P$ is called the
\emph{vertex set} of the polytope $P$. 

\end{definition}

\begin{definition}
 (H-polytope) An \emph{H-polytope} is a bounded set which is the intersection of a finite set of 
half-spaces, that is
\begin{equation}
 P = \bigcap_{i=1}^n \left\{x \in \mathbb{R}^d \mid \mathbf{a}_i^T \mathbf{x} 
\leq 
c_i \right\}
\end{equation}
where $\mathbf{a}_i \in \mathbb{R}^d$ are normal vectors of the hyperplanes and 
$c_i \in 
\mathbb{R}$ are displacements. If this description is minimal with respect to 
the number of half-spaces required to describe $P$, it is called 
\textit{irredundant}.
\end{definition}

It is a fundamental theorem of polytopes that these two representations are 
equivalent.
\begin{theorem}[Minkowski-Weyl Theorem {\cite[Thm. 1.1.]{PolyLec}}] Let $P \subset \mathbb{R}^d$. 
$P$ is an $H$-polytope if and only if $P$ is an $V$-polytope.
\end{theorem}
The \textit{affine hull} of a set of points is the dimension of the smallest affine subspace which 
contains all of the points The dimension of a polytope is the dimension of its affine hull. 
A $d$-polytope in $\mathbb{R}^d$ is said to be 
\textit{full-dimensional}. If the lowest dimensional affine subspace which contains 
$n$ points is $(n-1)$-dimensional, the points are said to be \textit{affinely 
independent}.

 The boundary 
of a polytope consists of lower dimensional polytopes called faces. 
\begin{definition}
 (Face) A \emph{face} is an intersection of a polytope with a hyperplane 
\begin{equation}
 F = P \cap \{x \in \mathbb{R}^d \mid \mathbf{a}\mathbf{x} = c\}
\end{equation}
where the hyperplane has to be \textit{valid}, that is, $\mathbf{a}\mathbf{x} 
\leq c$ holds for all points in $P$. 

\end{definition}
Both the empty set and $P$ are faces of $P$, they are given by $0x \le 1$ and $0x \le 0$, respectively.
The empty set is defined to have dimension $-1$.
A face that is not equal to all of $P$ is called a \textit{proper face}.
Some faces have special names: the 0-dimensional faces are called 
\textit{vertices}, the 1-dimensional faces are called \textit{edges}, the $(d-2)$-dimensional
faces are called \textit{ridges} and the 
$(d-1)$-dimensional faces are called \textit{facets}. In an irredundant 
half-space description each of the inequalities corresponds to a facet.
 The faces of a polytope 
form a lattice with order given by inclusion. This lattice is called the 
\textit{face lattice} of the polytope.

The faces of various dimensions 
are counted by the face vector.
\begin{definition}
 (Face vector) The \emph{face vector} of a $d$-polytope $f(P)$ is $(f_0, f_1, \dots, f_{d-1}) \in \mathbb{Z}^d$ 
where $f_i$ is 
the number of $i$-dimensional faces of $P$.
\end{definition}
The face vector satisfies the famous Euler-Poincare equation
\begin{equation}
 \sum_{i=0}^{d-1} (-1)^i f_i = 1 - (-1)^d.
\end{equation}
This is the only linear equation satisfied by the face vector. More 
constraints can be found by studying the \textit{flag vector}.

\begin{definition}
 (Flag vector) The \emph{flag vector} flag$(P)$ has as components the number of chains $f_{i_1,i_2,\dots i_k}, \{i_1,i_2,\dots, i_k \} \subset \{0,1,\dots, d-1\}$ for all
 $f_{i_1} \subset f_{i_2} \subset \dots \subset f_{i_k}$.
\end{definition}

For example, the entry $f_{0,1,3}$ counts the number of chains in the face 
lattice which have elements in dimensions $(0,1,3)$. This means a vertex contained in an
edge, which is contained in a 3-face. The complete flag vector has $2^d$ entries,
but many of them are redundant. For example, we always have $f_{0,1} = 2f_1$ since each edge
connects two vertices. In particular, in 4 dimensions where we will be mostly working in,
knowing the entries $(f_0, f_1, f_2, f_3, f_{03})$ is enough to figure out the complete flag vector.

Now that we have defined what a polytope is, we need a criterion to decide which polytopes 
count as ``the same". There are two ways to do this, a combinatorial one and a geometric one.
Two polytopes $P\subset \mathbb{R}^d, Q \subset \mathbb{R}^e$ are 
\textit{affinely isomorphic} if there is an affine map $f: \mathbb{R}^d 
\rightarrow \mathbb{R}^e$ which is a bijection between the points of the 
polytopes. A more relaxed equivalence relation is combinatorial equivalence. Two 
polytopes are \textit{combinatorially equivalent} if their face lattices are isomorphic 
i.e.\ there is an bijection between the elements of the lattices which preserves
the inclusion relation. Two combinatorially equivalent polytopes are said to have the 
same \textit{combinatorial type}. Any two affinely isomorphic polytopes are combinatorially
equivalent.

% Each polytope has a dual polytope associated to it. Its face lattice is the 
% lattice of the original polytope ``upside down''. Hence its $i$-faces 
% correspond to the $(d-i)$-dimensional faces of the original. The dual of the 
% dual is the original polytope. 

Two polytopes are said to be \emph{dual} to each other if there is an inclusion 
reversing bijection between their face lattices. Clearly, the dual of a 
$d$-polytope is another $d$-polytope. The fact that every polytope has a dual 
can be seen from the \textit{polar set}. Let $A\subset \mathbb{R}^d$, then the 
polar set $A^*$ is defined by
\begin{equation}
 A^* = \{y \in \mathbb{R}^d \mid \langle x,y\rangle \leq 1 \, \forall  x  \in A 
\} 
\end{equation}
If $A$ is a convex compact set with $0 \in$ int $A$, so is $A^*$. In 
particular, if $A$ is a polytope containing $0$ in the interior, $A^*$ is dual 
to it. 


We are interested in studying the graphs of polytopes. The most theoretically 
pleasing way to define them is to consider them a special class of 
\textit{polytopal complexes}.
A polytopal complex is a collection of polytopes that fit together nicely.
More precisely:
\begin{definition}
 (Polytopal complex) A polytopal complex $\mathfrak{C}$ is a finite set of 
polytopes in some $\mathbb{R}^d$ with the following properties:
\begin{enumerate}
 \item The empty polytope $\emptyset \in \mathfrak{C}$.
  \item (Downward closure) If $F \subset P$ is a face of $P$ and $P \in 
\mathfrak{C}$, then 
$F \in \mathfrak{C}$.
\item The intersection $P\cap Q$ of $P,Q \in \mathfrak{C}$ is a face of both.

\end{enumerate}

\end{definition}

The \emph{dimension} of a polytopal complex dim($\mathfrak{C}$) is the largest dimension
among the polytopes contained in it. Its \textit{underlying set} $|\mathfrak{C}|$ is the union of all of 
its polytopes $  |\mathfrak{C}| \coloneqq \cup_{F\in \mathfrak{C}} F.$

% Examples of polytopal complexes include the polytope itself and its 
% \textit{boundary complex} which is the collection of all proper faces of $P$. 
Any polytope $P$ and its faces form a polytopal complex. Another common complex is the 
\textit{boundary complex} of a polytope
$P$. The boundary complex is the collection of proper faces of $P$.
The $k$-skeleton of a polytope $P$ is the collection of all faces of dimension $\leq 
k$. Hence the $d$-skeleton of a $d$-polytope is the polytope and its faces, while the 
boundary complex is the $(d-1)$-skeleton. The graph of the polytope is its 
1-skeleton and it is denoted by $G(P)$.

After all these theoretical constructions, let's have a look at some concrete examples
of polytopes.
In any dimension, the polytope with the fewest vertices is the simplex. 
\begin{definition}
 (Simplex) A $d$-simplex is the convex hull of $d+1$ affinely independant 
points.
\end{definition}
Any two simplices of the same dimension are affinely isomorphic.
The faces of a simplex are lower dimensional simplices and any $k\le d+1$ vertices
of the simplex form a $(k-1)$-face. Therefore the number of $k$-faces is $\binom{d+1}{k}$
and the total number of faces is $2^{d+1}$.
 A polytope whose facets 
(but not necessarily the polytope itself) are simplices is called 
\textit{simplicial}. The dual of a simplicial polytope is called 
\textit{simple}. The simplex has the honor of being the only simple and simplicial polytope in dimensions three and greater.
The vertices of a simple $d$-polytope are contained in exactly 
$d$ edges and $d$ facets. These are also sufficient conditions for simplicity. This means that the graph of a simple polytope is $d$-regular.

As simplices are the least complicated polytopes, it is often useful to break up a polytope into
a union of simplices. A \emph{triangulation} of a polytope $P$ is a polytopal complex with underlying set
$P$ where all the polytopes are simplices.


Another important polytope is the $m$-cube.

\begin{definition}
 The $m$-cube $Q^m \in \mathbb{R}^m$ is a simple polytope defined by the inequalities
\begin{equation}
 0 \leq x_i \leq 1 
\end{equation}
for $i \in 1,\dots, m$. Equivalently, it is the convex hull of all points of the form $v \in \{0,1\}^m$.
\end{definition}
From the vertex description we see that the $m$-cube has $2^m$ vertices. A vertex $v$
is incident to $m$ edges, so the $m$-cube is a simple polytope. The neighbors of $v$ are
the vertices which agree in precisely $m-1$ coordinates.

From the halfspace description we see that there $2m$ facets. The facets (and therefore lower dimensional faces)
are also combinatorial cubes. Polytopes with this property are called \textit{cubical}. 

The non-empty faces of an $m$-cube can be characterized by whether the $k$th coordinate 
is required to be 0 or 1 or whether it can take all the values in $[0,1]$. Denoting the latter case by $*$, we see
that the faces correspond to vectors $F \in \{0,1,*\}^m$ plus the empty face. The number of $*$-entries 
is equal to the dimension of the face. Hence there are $\binom{m}{k} 2^{m-k}$ $k$-faces and $3^m +1$ faces in total.

There is an interesting connection between the simplex and the cube. If we draw a diagram
of the face lattice of a $d$-simplex by connecting each $k$-face with the $(k+1)$-faces containing it,
we end up with the graph of a $(d+1)$-cube.

Now that we have some polytopes to work with, we will present several ways to turn these polytopes into new ones.
Vertex truncation is one such method.
We will use it extensively.
\begin{definition}
(Vertex truncation) Let $P$ be a polytope with vertex set $V$. Truncating a vertex $v \in V$ means adding
a new inequality to the halfspace description, which is violated by $v$ but strictly valid for all the others vertices in $V$.
\end{definition}
All the polytopes formed this way are combinatorially equivalent (but generally not affinely isomorphic). 
For each face intersected by the hyperplane, 
a face of one dimension lower is added to the face lattice. The new facet is 
called the \textit{vertex figure} of the vertex $v$. The vertex figure's combinatorial type 
depends only on the choice of vertex, not on the inequality. Vertex cuts can be generalized to cutting of 
faces of higher dimensions by requiring the added inequality to be violated by 
all the vertices in the face. 

The dual operation of truncation is called stacking. What we mean by this is is that if we cut off vertex $v$
and take the dual of the resulting polytope is equivalent to first taking the dual and then stacking on the
corresponding facet.
Stacking means adding a new vertex to the polytope which satisfies all the 
inequalities in the irredundant description with strict inequality except the 
ones corresponding to the face being stacked on.

Another way to produce new polytopes is taking prisms. This operation takes a 
$d$-polytope $P$ in $\mathbb{R}^d$ and produces a $(d+1)$-polytope prism$(P)$ in 
$\mathbb{R}^{d+1}$. 
\begin{definition}
 A prism is the convex hull of a copy of $P$ embedded on the hyperplane 
$x_{d+1} = 0$ and another copy on the the hyperplane $x_{d+1} = 1$.
\end{definition}

 The new polytope prism$(P)$ has 2 copies of 
each face $F$ of $P$ at the top and bottom and for each face $F$ a face isomorphic to 
prism$(F)$ in the middle. Two vertices in $G(\text{pyr}(P))$ are adjacent if and only if
they are copies of the same vertex or if they are in the same copy and are 
adjacent in $P$. Because taking the prism increases the degree of each vertex by one, prism$(P)$ 
is a simple polytope if and only if $P$ is a simple polytope. We can easily calculate
the face vector of prism$(P)$ from the face vector of $P$.
\begin{align}
& f_0(\textrm{prism}(P)) = 2f_0(P) \\
& f_i(\textrm{prism}(P)) = f_{i-1}(P) + 2f_i(P) \quad 1 \le i \le d
\end{align}


The $n$-cube can be defined as the $n$-fold 
prism over the zero dimensional polytope (a point). 

The dual construction to prism 
is the bipyramid. 
\begin{definition}
 (Bipyramid) Let $P\subset \mathbb{R}^d$ be a $d$-polytope with zero in the 
interior. Then the bipyramid bipyr$(P)$ is the convex hull of $P$ embedded on 
the hyperplane $x_{d+1} = 0$ and the points $(0,0,\dots, 1)$ and $(0,0,\dots, 
-1)$.
\end{definition}


\section{Graphs}

Graph theory is a subject in its own right, with connections to all branches 
of mathematics. We neither can nor wish to give a complete introduction, but
rather focus on the parts necessary for our own study. More information on graphs
and their separators can be found in \cite{spectra} and \cite{Diestel}.

\begin{definition}
(Graph) A \emph{graph} is an ordered pair of sets $G=(V,E)$. The elements of $V$ are called 
\emph{vertices}. The elements of $E$ are 2-element subsets of $V$. They are called 
the \emph{edges} of the graph. 
\end{definition}
A graph is often visualized by connecting dots (vertices) with lines (edges) on a plane. 
If such a drawing can be made without any lines intersecting, the graph is called
\emph{planar}.
We consider only simple graphs, i.e.\ graphs without loops or parallel 
edges.

If $(u,v) \in E$, the vertices $u$ and $v$ are said to be \textit{adjacent}. 
The vertices $u$ and $v$ are \textit{incident} to the edge $(u,v)$. If there is an edge
between every pair of vertices, the graph is \textit{complete}.
\begin{definition}
 (Path) A sequence of edges $(u_1, u_2), (u_2, u_3), \dots (u_{k-1}, u_k) \subset E$ 
is called a \emph{path} between the vertices $u_1$ and $u_k$ if all the $u_i$ are 
distinct.
\end{definition}
If there is a path between every pair of vertices, the graph is said to be 
\textit{connected}. Otherwise it is \textit{disconnected}. A disconnected graph 
consists of several \textit{connected components}.


The \textit{degree} of a vertex is the number of edges it is incident to. 
If all the vertices in the graph have degree $d$, the graph is called 
\emph{$d$-regular}. A graph is called is \emph{$k$-vertex-connected} if removal of fewer than 
$k$ vertices does not disconnect the graph. Edge connectivity is defined in an 
analogous manner. Steinitz' theorem states that the graphs of 
3-polytopes are precisely all planar, simple, 3-vertex-connected graphs. The vertex connectivity
of a graph is at most as large as the minimum degree of the graph. For planar graphs the minimum degree
is at most 5, so we can disconnect the graph by removing at most 5 vertices. However, this
often means that one of the components is very small. A more nuanced view is 
provided 
by separators:

\begin{definition}
(Separator) \emph{A (vertex) separator} with separation constant $0<c<1/2$ is a 
partition of the vertices of a graph into 
sets $(A,B,C)$  with $cn \le |A| \le |B| \le (1-c)n$ so that there is no edge between 
a vertex in $A$ and a vertex in $B$. The \textit{size} of the separator
separator is $|C|$.  
\end{definition}

Another way to think about separators is to find a set of given size with as small
as possible \textit{neighborhood}.

\begin{definition}
 (Neighborhood) Let $S \subset V$ be a set of vertices. Then its \emph{neighborhood} 
$N(S)$ is the set $\{v \in V\setminus S \mid (u,v) \in E, \textrm{for some } u \in S\}$. 
\end{definition}

Separators can also be defined by removing edges instead of vertices. However, 
for regular graphs the sizes of these separators differ by a constant multiple,
so we ignore the distinction.
We are generally interested how the size of the smallest separator behaves as a 
function 
of the number of vertices for some fixed graph family.


The optimal upper bound on the size of minimal separators in the graphs of 3-polytopes was established by 
Lipton and Tarjan \cite{LiTa} in 1979. They showed that planar graphs have 
separators of size $O(\sqrt{n})$ for separation constant $\nicefrac{1}{3}$. This is 
the best possible result since an ($m\times m$)-grid requires $m$ vertices to be 
removed to be separated. However, in higher dimensions such results are 
impossible due to the existence of \textit{neighborly polytopes}.

\begin{definition}
 A \emph{$k$-neighborly polytope} on $n$ vertices is a $d$-polytope with the $k$-skeleton of an $(n-1)$-simplex.
A $\left\lfloor \frac{d}{2} \right\rfloor$-neighborly $d$-polytope is simply called \emph{neighborly}, as this is the maximal
neighborliness of a polytope which is not a simplex.
\end{definition}

The simplest and first discovered example of neighborly polytopes is the \textit{cyclic polytope} $C_d(n)$.
It is the convex hull of $n$ vertices which lie on the moment curve parametrized by $(t,t^2,\dots t^d) \subset \mathbb{R}^d$.
The graph of any neighborly polytope of dimension $d\geq 4$ is the complete graph, which can't be separated by removing vertices. Simple 
$d$-polytopes on the other hand have $d$-regular graphs and there better results could be 
possible. 

%An analogous class of polytopes which we will be using is called neighborly 
%cubical polytopes. 
%
%\begin{definition}
% A neighborly cubical polytope $\NC_d(m)$ is a cubical $d$-polytope with the 
%$\left\lfloor \frac{d}{2} 
%\right\rfloor$-skeleton of a $m$-cube.
%\end{definition}
%
%They were first discovered by Joswig and Ziegler. Sanyal and Ziegler \cite{Z102}
%generalized the construction to show that each $(d-2)$-dimensional neighborly 
%polytope on $(m-1)$ vertices can be used to define a 
%different combinatorial type of $\NC_d(m)$. 
%
%\begin{conjecture}
% (Kalai 1991 \cite{Ka2}, corrected version 2004 \cite{kalai04:_polyt}) The 
%graph of a simple $d$-polytope can be separated by removing 
%$O(n^{(d-2)/(d-1)})$ vertices. 
%\end{conjecture}
%
%This reduces to the planar separator theorem in the 3-dimensional case. 


\chapter{Lower bound on separators of simple polytopes}
\label{ncpconstruction}

In this chapter we construct polytopes that disprove Conjecture \ref{simpleconjecture} and
strongly affirm Thurston's claim about separators of dual graphs of triangulations. We
calculate a lower bound on the size of the minimal separators in two ways. Both proofs are for the
4-dimensional case, but they easily generalize to higher dimensions. The first proof requires
no advanced techniques, but the bound achieved is not sharp. For the second proof,
we restrict ourselves to a subset of the neighborly cubical polytopes where we know the 
combinatorics completely. This combined with a lemma proven by Sinclair to study
the mixing properties of Markov chains gets us the sharp bound. The main result of this
thesis is the following theorem.


\begin{theorem}
\label{maintheorem}
 Let $d\geq 4$. There exist a sequence of $d$-dimensional simple polytopes with
separators of size $\Omega(n/\log n)$.
\end{theorem}

\section{Doubly truncated neighborly cubical polytopes}
\label{ncp}

We start by introducing neighborly cubical polytopes. They are the block from which we carve our simple polytopes.

\begin{definition}
A neighborly cubical polytope $\NC_d(m)$ is a cubical $d$-polytope whose $\left\lfloor \frac{d}{2} 
\right\rfloor$-skeleton is isomorphic to the $\left\lfloor \frac{d}{2} 
\right\rfloor$-skeleton of the $m$-cube.
\end{definition}

Such polytopes were first discovered by Joswig and Ziegler \cite{Z62}. Sanyal and Ziegler \cite{Z102}
generalized the construction to show that each $(d-2)$-dimensional neighborly 
polytope on $m-1$ vertices can be used to define a 
different combinatorial type of $\NC_d(m)$. As the boundary complex of any $\NC_d(m)$
is a subcomplex of the boundary complex of the $m$-cube, its proper faces correspond to vectors
in $\{0,1,*\}^m$, or equivalently, $\{-,+,*\}^m$.


Our construction starts from a neighborly cubical 4-polytopes $\NC_4(m)$. 
Their graph is that of an $m$-cube. The minimal separators of $m$-cubes were determined by Harper 
\cite{Harp}.
While hypercubes are simple polytopes, $\NC_4(m)$ are not since each vertex is 
incident to $m>4$ edges. However, we can ``simplify" them by 
truncating the vertices and the edges.

 

\begin{lemma}
(See also Ewald \& Shephard \cite{EwSh}) A $d$-polytope can be made simple 
by cutting off the vertices, then the original edges and so on up to $(d-2)$-faces.
\end{lemma}

\begin{proof}
This is easier to see in the dual case. The dual operation of cutting a vertex 
is stacking on a facet. By stacking on all facets one after the other, we 
remove all the original facets and introduce new ones, which are pyramids over 
$(d-2)$-faces (ridges). The original $(d-2)$-faces are preserved and we can stack on 
them, removing them and the facets created in the previous operation. Every 
iteration of this operation removes all the facets created in previous 
iterations and creates new facets which are iterated pyramids on successively 
lower dimensional faces. Once we have done this $d-2$ times, the iterated 
pyramids are over edges and hence they are simplices. This means that the facets are 
simplices, so the polytope is simplicial and its dual is simple.
\end{proof}

Therefore, by cutting of the vertices and then the original edges, we can turn $\NC_4(m)$ into a simple 
polytope. Let us see what this polytope looks like. 

The starting polytope $\NC_4(m)$ has the flag vector 
\begin{align*}
 \textrm{flag}(\NC_4(m)) &:= (f_0, f_1, f_2, f_3; f_{03}) \\
	&= (2^m, m2^{m-1}, 3(m-2)2^{m-2}, (m-2)2^{m-2}; 8(m-2)2^{m-2}) \\
	&= (4, 2m, 3(m-2), m-2; 8(m-2))2^{m-2}
\end{align*}

The first two entries follow from the fact that the graph is the same as that 
of the $m$-cube. Since the facets are 3-cubes, they 
consist of 6 squares, and each of these squares is an intersection of two 
3-cubes. Hence we know that $6f_2 = 2f_3$ and we can calculate these values with 
the Euler-Poincar\'{e} formula. The final entry follows from the fact that the facets are 
3-cubes and hence consists of 8 vertices.

We denote the polytope formed by truncating all the vertices by $\NC_4(m)'$.
Truncating the vertices means removing all of the original vertices, but 
adding 2 new vertices for each edge, one at each end of the edge. The facets of this new 
polytope come in two types:

\begin{itemize}
 \item The $(m-2)2^{m-2}$ original facets, which are cubes whose vertices are 
cut off. These are simple polytopes with $f$-vector (24,36,14).
 \item The $2^m$ new facets which are the vertex figures. The facets of these 
facets are vertex figures of 3-cubes and hence triangles. Therefore these 
new facets are simplicial. Their $f$-vector is ($m$, 
$3m-6, 2m-4)$.
\end{itemize}

With this information of the facets we can work out the flag vector of $\NC_4(m)'$:
\begin{equation}
 \textrm{flag}(\NC'_4(m)) = (4m, 14m-24, 11m-22, m+2; 28m-24)2^{m-2}.
\end{equation}

By cutting of all of the original (but shortened) edges in $\NC_4(m)'$, we arrive at 
$\NC_4(m)''$. It has three types of facets:

\begin{itemize}
 \item $(m-2)2^{m-2}$ simple polytopes which correspond to the facets of 
$\NC_4(m)$, but whose vertices and edges have been cut. Their $f$-vector is 
(48, 72, 26).
\item $m2^{m-1}$ prisms over polygons with $3$ to $m-1$ sides. They come from 
the edges of the original polytope and the number of sides depends on the degrees of 
vertices incident to the edge.
\item $2^m$ simple polytopes which are the truncated vertex figures of 
$\NC_4(m)$. Their $f$-vector is $(6m - 12, 9m-18, 3m-4)$.
\end{itemize}

Using the knowledge of the facets we can work out the flag vector of 
$\NC_4(m)''$:

\begin{equation}
 \textrm{flag}(\NC_4(m)'') = (24m-48, 48m-96, 27m - 46, 3m+2; 28m -48)2^{m-2}
\end{equation}

Since $2f_0 = f_1$, we can verify that $\NC_4(m)''$ is indeed a simple 
polytope. In the next section we will work out its separators.

\section{Lower bound, straightforward approach}

In this section we will show an easy proof for a lower bound on the separator. It
is enough to disprove Conjecture \ref{simpleconjecture}, but not enough to verify Thurston's
claim.

We are interested in what the graph $G(\NC_4(m)'') = G''_m$ looks like. Its 
most striking feature is its similarity to the graph of the hypercube, $Q_m$. 
$Q_m$ is a graph on $2^m$ vertices with $m2^{m-1}$ edges. The vertices of $Q^m$
are labelled by vectors in $\{0,1\}^m$. $G''_m$ 
similarly consists of $2^m$ groups of $(6m-12)$ vertices which we call 
\textit{clusters}. Each cluster is a 3-regular graph and is connected to its
$m$ neighbors by 3 to $m-1$ edges. We will use this similarity to 
prove a bound on the size of separators of $G''_m$. Let us therefore first take a 
look at the separators of $Q_m$.

The problem of finding the sets of size $k$ with minimal size neighborhood in a graph $G$ is 
called the \textit{discrete isoperimetric problem}. It is a generalization of 
the minimum separator problem as the size of the set is fixed but not required to be a linear fraction
of the vertices. 
Harper \cite{Harp} solved the problem on the hypercube in the 1960's. The 
answer turns out to be approximately Hamming balls.

\begin{definition}
 (Hamming ball) Let $a,b \in \{0,1\}^m$. The Hamming distance between $a$ and 
$b$ is the number of coordinates where $a$ and $b$ differ. The Hamming ball 
$B(x,r) \subset \{0,1\}^m$ is the set of elements at Hamming distance $\leq r$ from $x$ for $x \in \{0,1\}^m$.
\end{definition}

A Hamming ball $B(x,r) \subset Q_m$ contains
\begin{equation}
 \sum_{i=0}^r {m \choose i}
\end{equation}
vertices. If $k$ can be written in this form for some $r \in \mathbb{N}$, any Hamming ball with radius $r$ has minimal size neighborhoods 
among the $k$-element sets of $Q_m$. 
If $k$ can't be expressed in this form, let $r$ be maximal so that 
the sum is $\leq k$ and include some vertices with distance $r+1$ to the set. 
However, not all such choices is minimize the neighborhood. Suppose $x$ is the vertex labeled with 
all zeros. Then the extra vertices are those which come first in the
\textit{lexicographic order} of vertices with $r+1$ ones and $m-r-1$ zeros.
\begin{definition}
 (Lexicographic order) Let $a,b \in \{0,1\}^m$. Then $a<b$ if for some $i$ the first $i$ 
elements of $a$ and $b$ are the same but $a_{i+1} < b_{i+1}$.
\end{definition}
Hence there exist an increasing sequence of sets with minimal vertex neighborhood. Increasing sequence means that
the set with size $k$ is contained in the set with size of $k+1$. The order in which the 
vertices are added as $k$ increases is also called \textit{graded lexicographic 
order}. Let us now calculate how the size of the neighborhoods behaves asymptotically.

Let $k = c2^m$ for some $0<c<\nicefrac{1}{2}$ and let $m \rightarrow \infty$. By the central limit theorem, 
the set with minimal neighborhood contains a Hamming ball with radius $r \approx m/2$. This is because the middle layers contain most of the vertices.
The asymptotics are fairly easy to work out, (see for example \cite{Spencer:Asymptopia}):
\begin{equation}
\binom{m}{(m+i)/2} \sim \sqrt{\frac{2}{\pi m}} 2^m e^{-i^2/2m}  
\end{equation}
for $i \in o(m^{2/3})$. For fixed $i$ this is $\Theta(2^m/\sqrt{m})$.
%By Stirling's formula,

%\begin{equation}
% {m \choose m/2} = \Theta \left(\frac{2^m}{\sqrt{m}}\right)
%\end{equation}

We will use the fact that separators of $Q_m$ are large to show that the separators of $G''_m$ are not small.
Let $f: G_m'' \rightarrow Q_m$ 
be the map that sends all the vertices of a cluster to the corresponding vertex 
in the cube graph. 
Let $(A,B,C)$ be a separator in $G''_m$. Then we can partition
$Q_m$ to sets $(a,b,c)$ using $f$ and the separator $(A,B,C)$. If the preimage of a vertex $v$ is comply contained in A, we 
put $v \in a$ and do the same for $B$ and $b$. Otherwise, $v \in c$. Since the clusters are connected 
graphs, any cluster which contains vertices in $A$ and $B$ must also contain 
vertices in $C$. This means that the preimage of a vertex $v \in c$ contains vertices in $C$. Hence $|c| \leq |C|$. 
Now, no vertices with labels $a$ and $b$ are adjacent since this would imply 
that a cluster whose vertices all belong to $A$ is adjacent to a cluster whose vertices all belong to 
$B$. This is impossible since we assumed that $(A,B,C)$ is a separator. Therefore
\begin{enumerate}
 \item $(a,b,c)$ is a separator of $Q_m$, or
\item one or both of $a,b$ is small.
\end{enumerate}
In the first case, $c$ contains at least the neighborhood of $a$ and $b$ and 
hence its size is bounded by the isoperimetric inequality. Therefore $c$ has 
$\Omega(2^m/\sqrt{m})$ vertices. Hence $C$ contains at least $\Omega(n/(\log 
n)^{\frac{3}{2}})$ vertices.

In the latter case, $c$ must contain a linear number of vertices. This means 
that the set $C$ must also be large.  

\section{Sharp lower bound}

To prove the sharp bound, we need some extra tools. Let $G= (V,E)$ be a simple graph and
 $S\subset V$. The \textit{edge boundary} $\delta(S)$ is defined as the set of edges with one end
in $S$ and the other in $V\setminus S$. The \textit{edge expansion} $\mathcal{X}(G)$ is then defined
as 
\begin{equation}
\mathcal{X}(G) \coloneqq  \min \left\{ \frac{ |\delta(S) |}{ |S |}, S \subset V, S\neq \emptyset, |S| \leq \frac{ |V |}{2}  \right\}.
\end{equation} 
This quantity gives information on mixing properties of Markov chain on $G$. We study it since
a large edge expansion on a regular graph also implies a large separator (see \ref{separatorsize} below).
The edge expansion can be estimated by ``canonical paths" method by Sinclair \cite{Sinclair}. 
A good introduction, with applications to the graphs of polytopes, is given by Kaibel \cite{Kaibel}. 

\subsection{Sinclair's canonical paths}

To estimate the edge expansion of a graph $G$, we specify a path between every (ordered) pair of vertices $(s,t) \in V \times V$.
The intuition behind the method is that in graphs with small edge expansion, there necessarily exist edges which are ``congested," that is, there
are many paths going through them. To make this intuition precise, let $\phi: E(G) \rightarrow \mathbb{N}$ count the number of paths
that traverse a given edge. The edge expansion can be bounded 
in terms of
\begin{equation}
\phi_{\max} \coloneqq \max \{ \phi(e), e \in E(G) \}.
\end{equation}


\begin{lemma}[Sinclair's lemma]
\label{Sinclair}
Let $\phi_{\max}$ be as defined in previous paragraph. Then the edge expansion of $G$ satisfies 
\begin{equation}
\mathcal{X}(G) \ge \frac{n}{2 \phi_{\max}}.
\end{equation}
\end{lemma}

\begin{proof}
Let $\phi(\delta(S))$ be the sum of $\phi(e)$ over all the edges in $\delta(S)$. Every path that has one end in $S$
and the other in $V\setminus S$ needs to use one of these edges, so we have $\phi(\delta(S)) \ge   |V\setminus S | |S |$.
On the other hand, $\phi_{\max}$ is the global maximum over all the edges of the graph, so we have $\phi(\delta(S)) \le  |\delta(S) |\phi_{\max}$.
Hence for any $|S| \le \frac{ |V |}{2}$, we have
\begin{equation}
\label{sepbound}
 \mathcal{X}(G) \ge \frac{|\delta(S)|}{|S|} 
                \ge \frac{\phi(\delta(S))}{\phi_{\max}\,|S|}
                \ge \frac{|S|\,|V{\setminus}S|}{\phi_{\max}\,|S|}
                 =  \frac{|V{\setminus}S|}{\phi_{\max}}
				\ge \frac{n}{2\,\phi_{\max}}.
\end{equation}
\end{proof} 
\subsection{Relating edge expansion and separator size}
Now we will show that large edge expansion implies a large separator.
\begin{lemma}
\label{separatorsize}
Let $G$ be a $d$-regular on $n$ vertices with edge expansion $\mathcal{X}(G)$.
Then its separators are of size at least
\begin{equation}
	\frac{c}{d}\mathcal{X}(G)n = \Omega(\mathcal{X}(G)n),
\end{equation}
for fixed $d$ and $n\rightarrow \infty$. Here $c$ is the constant from the definition of a separator.
\end{lemma}
\begin{proof}
Let $G$ be as defined above and let $(A,B,C)$ be a separator of $G$ with $|B| \ge |A| \ge cn$.
As there are no edges between $A$ and $B$, any edge in the set $\delta(A)$ has its other end in
$C$. This means that $\delta(A)$ contains at least $|A|\mathcal{X}(G)$ edges. Since $G$ is $d$-regular,
the size of $C$ is at least $\mathcal{X}(G)|A|/d \ge \mathcal{X}(G)cn/d = (c/d)\mathcal{X}(G)n$.  
\end{proof}

\subsection{Cyclic neighborly cubical polytopes}

For the sharp bound, we will restrict to \textit{cyclic neighborly cubical polytopes} 
 $\NC^c_d(m)$.
This was the combinatorial type constructed by Joswig \& Ziegler \cite{Z62}. They
are also a result of using the construction method of Sanyal \& Ziegler \cite{Z102}
on cyclic polytopes with vertices in standard order, which justifies the name.  According
to their analysis, the vertex figures are combinatorially equivalent to a pyramid over
a triangulation of the cyclic polytope $C_{d-2}(m-1)$. Since the graph $G_m$ of $\NC^c_d(m)$
is isomorphic to that of the $m$-cube, we will label the vertices by vectors in $\{-,+\}^m$ and
edges by vectors in $\{-,+,*\}^m$ with exactly one $*$-entry. The index of the $*$-entry is
called the \textit{direction} of the edge. There are $2^{m-1}$ edges in each direction.
The combinatorics of $\NC^c_d(m)$ are fully described by the Cubical Gale Evenness Criterion.
This criterion is fairly complicated even in the 4-dimensional case, but we do not need the 
complete description.

\begin{theorem}
[Part of the Cubical Gale Evenness Criterion, for $d=4$ {\cite[Thm.~18]{Z62}}] 
The facets of the cyclic neighborly polytope $\NCC_4(m)$ are given by vectors in
$\{-,+,*\}^m$ with exactly three $*$-entries. 

If the first component of a vector in $\{-,+,*\}^m$ is $*$, the vector corresponds to a facet 
of $\NCC_4(m)$ if and only if the rest of the vector satisfies the Gale Evenness Criterion, 
that is, if between any two non-$*$-entries there is an even number of $*$-entries.
Equivalently, this happens if in the rest of that vector the two $*$-entries are
cyclically adjacent.
\end{theorem}

 This shows that for any vertex $v$, the edges in directions $i$ and $i+1$ span a 2-face, because
the vector $(*,v_2, \dots, v_{i-1}, *, *, v_{i+2}, \dots, v_m)$ corresponds to a facet of $\NCC_4(m)$
for $i\le 2 < m$. As the 2-faces correspond to edges in the vertex figure, this means that the vertices 
of the vertex figure at $v$ are cyclically connected in a consistent way independent of $v$. In other words
there is a Hamiltonian cycle $(1,2,3,\dots, m)$ going through the vertices in the graph of each vertex figure.
This is also shows that the boundary complex of $\NCC_4(m)$ contains a copy of the polyhedral surface 
describe by Ringel \cite{ringel55:_ueber_probl_wuerf_wuerf} in 1957. This was also pointed out by
Ziegler \cite[Sect 3.]{Z100}. See also Joswig \& Rörig \cite{joswig:_neigh}.

Let us take a closer look at $\NCC_4(m)'$ first.  Each vertex in a "new" facet is labelled naturally by
the direction of the edge of $G_m$ it is incident to. Therefore we can label the vertices of $G'_m$
by $(v,i) \in \{-,+\}^m \times [m]$ where $v \in \{-,+\}^m$ is the label of the vertex of $\NCC_4(m)$ which has been truncated and $i \in [m]$ 
is the direction of the edge of $\NCC_4(m)$ which has been cut. The vertex figures of $G'_5$ are illustrated in Figure \ref{figure:graphs}.

We divide the edges of $\NCC_4(m)'$ into three different classes. First of all, there are the "long" edges which correspond to the edges $\NCC_4(m)$.
They are incident to vertices $(v,i)$ and $(v',i)$ where $v$ and $v'$ only differ in coordinate $i$. There $m2^{m-1}$ such edges in total. Secondly, 
there are "medium" edges, which are between vertices $(v,i)$ and $(v,i+1\mod m)$. They form a Hamilton cycle $(1,2,3,\dots, m)$ on each cluster $v$. In total there are $m2^m$ medium edges. The rest of the edges are ``extra", they are between vertices $(v,i), (v,j)$ with $j-i \neq \pm 1 \mod m$.
There are $(2m-6)2^m$ ``extra" edges in total. The ``extra" and ``medium" edges together form the graphs of the vertex figures of $\NCC_4(m)$ which we call clusters. 

Truncating the long edges of $\NCC_4(m)'$ gets us the polytope $\NCC_4(m)''$. The ``medium" and ``extra" edges are shortened by this procedure, but are otherwise unchanged. The ``long" edges are replaced by their edge figures, which are prisms over $k$-gons with $3 \le k \le m-1$. Here $k$ is the degree of the incident vertices in their clusters, or equivalently their degree minus 1. We call the edges between these $k$-gons a \textit{parallel class of long edges}. The $k$-gon consist of ``short" edges. A picture of a cluster of $G''_5$ can be seen in the bottom figure of \ref{figure:graphs}.

The cycles of short edges naturally correspond to vertices of $G_m'$, so we extend the notation $\{-,+\}^m \times [m]$ to them. ``Medium" edges connect subsequent cycles $(v,i), (v,i+1 \mod m)$ while ``extra" edges connect non-subsequent cycles. 

The similarity of the graphs is apparent in their minor relations

\begin {equation}
G_m'' \rightarrow G_m' \rightarrow G_m
\end{equation}
We can get $G_m'$ by contracting the cycles of short edges in $G_m''$ and identifying the parallel classes of long edges 
to a single edge. $G_m$ is obtained from $G_m'$ by contracting all of the medium and extra edges.


\subsection{Canonical paths for $G''_m$}
\label{canpaths}

Now we will construct the canonical paths required for \ref{Sinclair} in $G''_m$. Our construction relies on the similarity
of the clusters and their cubical connections with each other.

Let $v_0$ be a vertex in the cycle of short edges $(v,i)$ and $w_0$ be a vertex in the cycle of short edges $(w,j)$. 
We will construct a path from $v_0$ to $w_0$ in the following way. Consider the coordinates in cyclic order $i, i+1, \dots m, 1, \dots,
i-1$ and for each coordinate do the following:

\begin{quote}
\textbf{Procedure P:}
Suppose the path we have constructed so far ends at vertex $u_0 \in (u,k)$.
\begin{compactitem}
\item If $u$ and $w$ differ in coordinate $k$, take the long edge from $u_0$ to $(u',k)$,
where $u$ and $u'$ differ only in coordinate $k$. Then, using only short edges and a medium edge,
move to the cycle of short edges labelled $(u', k+1)$.
\item Otherwise, $u$ and $w$ agree in coordinate $k$. Using only short edges and a medium edge,
move to the cycle $(u, k+1)$.
\end{compactitem}
\end{quote}

After performing this procedure at most $m$ times, we end up in the cluster $w$. After at most $m-1$ more
iterations, we end up in the cycle $(w,j)$. In the final iteration, we take at most $m-1$ steps in this cycle to arrive
at $w_0$.

We will now show that no edge is used more than $\Theta(m^2 2^m)$ times by considering the four cases of short, medium, long
and extra edges separately.

\afterpage{% 
\begin{figure}
	\input figure1.tex
\bigskip

	\input figure2.tex
\bigskip

	\input figure3.tex 
	
\caption{The graphs $\CG_m$, $\CG_m'$, and $\CG_m''$, for $m=5$:
	Local situation at one cluster.}
\label{figure:graphs}
\end{figure}	
\clearpage
}

\subsection{Long edges}

On average, a random pair of clusters differs in $m/2$ coordinates. Therefore half of the $n^2$ paths use long edges in direction $i$.
The procedure works independently of the value of the other coordinates, so each of the $2^{m-1}$ parallel classes in direction $i$ is used
equally often. Every parallel class is used by $(n^2/2) / 2^{m-1}$ paths and therefore each long edge is used at most this number of times. The number of vertices is $n \in O(m2^m)$, so this works out to be $O(m^2 2^m)$.

\subsection{Medium and short edges}

Every path we construct uses at most $2m$ iterations of procedure $P$. The total number of iterations over all paths is therefore at most
$2mn^2$.  Every iteration uses at most one long edge, one medium edge and short edges
on one cycle. The extra edges are not used at all. 

Our construction treats all coordinates and both values of the coordinates equally. This means that 
each set of "cycle of short edgess and a medium edge leaving it"
is used equally often. Moreover, these sets are disjoint. Therefore every short and medium edge is used at most $2mn^2 / (m2^m)$ times. 
As $n \in \Theta(m^2 2^m)$, this is works out to be $O(2mn^2 / m2^m) = O(m^2 2^m)$. 

\subsection{Wrapping things up}

In Section \ref{ncp} we showed that $G''_m$ has $(6m-12)2^m$ vertices. Let us
now combine our canonical paths with Sinclair's lemma \ref{Sinclair}:
\begin{equation}
\mathcal{X}(\CG_m'') 
\ge \frac{n}{2\,\phi_{\max}} 
\ge \frac{n}{2n^2/2^m}
  = \frac{2^m}{2n}
  = \frac{2^m}{2(6m-12)2^m}
  = \frac{1}{12(m-2)}. 
\end{equation}
With Lemma \ref{sepbound}, we get a bound on the size of minimal separators:
\begin{equation}
\frac c4 \mathcal{X}(\CG_m'')n 
\ge \frac{c\,n}{48(m-2)} 
 =  \Omega\Big(\frac n{\log n}\Big),
\end{equation}
as $n = (6m-12)2^m$ and thus $m=\Theta(\log n)$.

\section{Upper bound on minimal separators in $\NC_4(m)$}

The minimal separators of $G''_m$ have $O(n/\log n)$ vertices. Separators with this
many vertices are easy to construct. Pick a random coordinate direction $i$ and separate
the vertices into two sets: Set $A'$ where the coordinate $i$ is 0 and set $B'$ where the coordinate
is 1. Both sets have $(3m-6)2^m$ vertices and there are $O(2^m) = O(n/\log n)$ edges between them. For each of
these edges, pick one end point and move it to $C$. The remaining vertices in $A'$ and $B'$ form
the sets $A$ and $B$. This works for separation constant arbitrarily close to $\nicefrac{1}{2}.$

\chapter{Generalizations and possible larger separators}
\label{genes}

The sharp lower bound can be also proven for a larger class of neighborly cubical polytopes.
The polytopes constructed by Sanyal and Ziegler \cite{Z102} all contain the Hamiltonian
cycle we use to construct our paths. The triangulation of the clusters in $G'_m$ is described by
\cite[Thm 3.7.]{Z102}. For vertex $v$, it is a regular triangulation created by pushing 
and pulling vertices depending on the vertex label and the order of the vertices. 
The pushing and pulling only affects extra edges, so we end up with a Hamiltonian
cycle through the edges. However, this cycle is not necessarily $(1,2,3,\dots,m)$ but
rather some other fixed cycle on $[m]$.

This construction can also used the same bound in higher dimensions. There are two 
natural ways, using $\NC_d(m)$ i.e.\ a higher dimensional neighborly polytope, or 
taking prisms over $\NC_4(m)''$.  In the first case we have to do more cuts to 
make the polytope simple, resulting in a large number of vertices. Hence the 
second way is the correct one. 

Let $G^d_m$ be the graph of a $(d-4)$-fold prism over $\NC_4(m)''$ for $d\geq 4$. 
Any separator 
$(A,B,C)$ of $\NC_4(m)''$ can be turned into a separator of $G^d_m$ by setting 
$(A',B',C')$ to be the prisms over their respective sets. Since no vertices 
$a\in A, b\in B$ are adjacent and all the copies of a single vertex are in the 
same set, there are no edges between vertices in $A'$ and $B'$. This means that 
$G^d_m$ is at least as easy to separate as $\NC''_4(m)$. On the other hand, $G^d_m$ 
also has a cubelike structure and hence with the discrete isoperimetric 
inequality we get the same upper bound as for $\NC''_4(m)$. 

Similarly, the sharp bound can be extended to higher dimensions. A $(d-4)$ fold
prims over $\NCC_4(m)''$ adds $(d-4)$ extra parallel classes of long edges incident
to each vertex. We can amend our algorithm by taking a long edge in these directions 
if necessary and then proceeding in the usual way.

\section{Zig-Zag product}

Expander graphs are graphs which are as hard as possible to separate. This means that their
separators have $\Omega(n)$ vertices.
Even though most regular graphs of degree $d\ge 3$ are expanders \cite{KoBa1967}, 
constructing explicit examples is hard. Graphs of simple 4-polytopes on the other hand
are few and far between. While our $\NC_d(m)$ are not expanders, they are not that  
far away from expanding either. The key idea in our construction is taking a polytope which
is a fairly good expander (an $m$-cube)
and modifying it so that it becomes a simple polytope (and therefore a regular graph). This
is done by replacing the vertices with 3-regular graphs. There is a systematic way of doing this operation
on graphs which guarantees good separation properties. It is called
the zig-zag product. The zig-zag product was introduced by Reingold, Vadhan and Wigderson in 2000 
\cite{892006}. 

The zig-zag product takes a large regular graph $G$ and a small regular graph $H$ and produces a new
graph $G \circ H$. More precisely, the graph $G$ is a $D$-regular graph on $N$ vertices, while $H$ 
is a $d$-regular graph on $D$ vertices. Their product is then a $d^2$-regular graph on $D\times N$ vertices. 
The important part is that if $G$ and $H$ are good expanders, then $G \circ H$ is too. Before we can give
the formal definition of zig-zag product, we need to define \textit{rotation maps}.
\begin{definition}
Let $G = (V,R)$ be a $d$-regular graph. The edges incident to vertex $v$ are labeled with numbers $1,\dots, d$. 
The rotation map $\textrm{Rot}_{G} : V \times [d] \rightarrow V \times [d]$
returns for the $(v,i)$ pair the vertex $w$ which across the $i$th edge from $v$, and the label of this edge at $w$.
\end{definition}
Note that in the previous definition, each edge gets two labels, one for each endpoint. Not all graphs have rotation
maps so that the label at both ends is the same.

Using rotation maps of $G$ and $H$ we can now define the rotation map of $G \circ H$. 
\begin{definition}
Let $G$ be a $D$-regular graph on $N$ vertices and $H$ is a $d$-regular graph on $D$ vertices.
The rotation map $\textrm{Rot}_{G\circ H}$ is found with the following procedure:
\begin{flalign*}
&\textrm{Rot}_{G \circ H}((v,a),(i,j)): \\
&1. \textrm{Let} (a', i') = \textrm{Rot}_H (a,i) \\
&2.   \textrm{Let} (w, b') = \textrm{Rot}_G (v,a') \\
&3.  \textrm{Let} (b, j') = \textrm{Rot}_H (b',j) \\
&4. \textrm{Output} ((w,b), (j',i')
\end{flalign*}
\end{definition}
Having the edges defined locally is an advantage in computational applications. While the zig-zag product
necessarily produces good expanders, the results may not be polytopal graphs, even if the starting graphs are
polytopal. One stumbling block is that a polytopal graph of $d\ge 4$ must have many planar subgraphs and short
cycles, which correspond to the 3- and 2-faces. Both of these are easy to separate, but as our truncated neighborly cubical
polytopes show, the separator might still have to be large. Another problem is that the resulting graphs have degree $d^2$. While this
allows for creation of 4-dimensional polytopes, the only 2-dimensional polytopes are polygons they might not possess sufficient variety
to create expander graphs. The next feasible dimension is 9, but there we have already lost most of our geometric intuition.
 It is also possible that some choices of the rotation map produce polytopal graphs while others don't.

 As shown by the neighborly polytopes, a given graph can be the graph of polytope of various dimensions. 
Luckily, simple polytopes don't have this problem and there are even algorithms to solve this problem \cite{KALAI1988381} \cite{Achatz2000}. 
Our quick experiment by hand found a product of a tetrahedron and a triangle, which was not a polytopal graph.



\addtocontents{toc}{\vspace{\baselineskip}}

\cleardoublepage
\phantomsection
\addcontentsline{toc}{section}{Bibliography}

\bibliographystyle{siam}
\bibliography{separators}

\selectlanguage{ngerman}


\phantomsection
\chapter*{Zusammenfassung}
\header{Zusammenfassung}
\addcontentsline{toc}{section}{Zusammenfassung}

\small{
Richard Stanle.}




\phantomsection
\chapter*{Selbstst\"andigkeitserkl\"arung}
%\header{Introduction}
\addcontentsline{toc}{section}{Selbstst\"andigkeitserkl\"arung}

Gem\"a{\ss} \S 7 (4) der Promotionsordnung des Fachbereichs Mathematik und Informatik der Freien Universit\"at Berlin versichere ich hiermit, dass ich alle Hilfsmittel und Hilfen angegeben und auf dieser Grundlage die Arbeit selbstst\"andig verfasst habe. Des Weiteren versichere ich, dass ich diese Arbeit nicht schon einmal zu einem fr\"uheren Promotionsverfahren eingereicht habe.

\vspace{0.8cm}
\noindent Berlin, den 

\vspace{1.6cm}
\noindent Lauri Loiskekoski


\end{document}

